\documentclass[12pt]{report}
\begin{document}
We introduce a stochastic version of the Lorenz `86 model by considering the system as a perturbation to a Hamiltonian flow (with small perturbation parameter $b$), and replacing the deterministic Hamiltonian with a stochastic one consisting of the sum of a `pendulum' $H_{1}$ and a `harmonic oscillator' $H_{2}$, each multiplied by a different Stratonovich Brownian motion. Our system is such that we may apply averaging techniques from \cite{XM} to find the behaviour of the energies on long time-scales $t/b$, $t/b^{2}$. To apply these we use action-angle coordinates and find that at scale $t/b$ the energies are approximately constant; we apply the second theorem to obtain the behaviour at scales $t/b^{2}$ and calculate the stochastic process to which the energies converge. The phase space of the pendulum is split into three regions according to the value of $H_{1}$; we find lower bounds on the exit time from each region. We also consider a simpler system than `L86: that of a stochastic pendulum with deterministic perturbation. First we use an `outwardly-directed' perturbation and look at exit times from each region of phase space; we apply the first averaging theorem and find that the average energy reaches the boundary of the inner region in finite time. Later we consider a Hamiltonian perturbation vector field and apply the second averaging theorem, calculating the general form of the limiting process in this case, and applying it to a particular example. 
\end{document}