\usepackage{amsfonts}
\usepackage{amsthm}
\usepackage[mathscr]{euscript}
\usepackage{amsmath}
\usepackage{amssymb}
\usepackage{tensor}
\usepackage{fullpage}
\usepackage{enumerate}
\usepackage{graphicx,caption}
\usepackage{float}
\usepackage{url}
\usepackage[hidelinks]{hyperref}
\usepackage{cleveref}
\usepackage[font=footnotesize,labelfont=footnotesize]{caption}
\usepackage[ansinew]{inputenc}
\usepackage{pbox}
\usepackage{cancel}
\usepackage{accents}
\usepackage{subcaption}
\usepackage{color}


\setlength\parindent{0pt}
\linespread{1.2}

\newcommand{\D}[2]{\frac{d #1}{d #2}}
\newcommand{\PD}[2]{\frac{\partial #1}{\partial #2}}
\newcommand{\eq}[1]{\begin{equation} #1 \end{equation}}
\newcommand{\eqlab}[2]{\begin{equation} #1 \label{eq:#2} \end{equation}}
\newcommand{\eqnn}[1]{\eq{#1 \nonumber}}
\newcommand{\algn}[1]{\begin{align} #1 \end{align}}
\newcommand{\algnn}[1]{\begin{align*} #1 \end{align*}}
\newcommand{\bra}[1]{\left\langle #1 \right|}
\newcommand{\ket}[1]{\left| #1 \right\rangle}
\newcommand{\braket}[2]{\langle #1 | #2 \rangle}
\newcommand{\Braket}[3]{\langle #1 | #2 | #3 \rangle}
\newcommand{\B}[1]{\mathbf{#1}}
\newcommand{\tvec}[2]{\left(\begin{array}{c} #1 \\ #2 \end{array}\right)}
\newcommand{\tvecs}[2]{\left(\begin{smallmatrix} #1\\#2\end{smallmatrix}\right)}
\newcommand{\twobytwomatrix}[4]{\left(\begin{array}{cc} #1 & #2 \\ #3 & #4\end{array}\right)}
\newcommand{\threevec}[3]{\left(\begin{array}{c} #1 \\ #2 \\ #3 \end{array}\right)}
\newcommand{\ph}{\varphi}
\newcommand{\TL}[1]{\tensor{\Lambda}{#1}}
\newcommand{\TG}[1]{\tensor{\Gamma}{#1}}
\newcommand{\TT}[1]{\tensor{T}{#1}}
\newcommand{\TA}[1]{\tensor{A}{#1}}
\newcommand{\Td}[1]{\tensor{\delta}{#1}}
\newcommand{\TAinv}[1]{\tensor{{(A^{-1})}}{#1}}
\newcommand{\M}{\mathcal{M}}
\newcommand{\N}{\mathcal{N}}
\newcommand{\Dc}{\mathcal{D}}
\renewcommand{\L}{\mathcal{L}}
\newcommand{\intp}[1]{\int{\frac{d^{3}\B{#1}}{(2\pi)^{3}}\:}}
\newcommand{\intE}[1]{\int{\frac{d^{3}\B{#1}}{(2\pi)^{3}}\frac{1}{2E_{#1}}\:}}
\newcommand{\intpi}[1]{\int{\frac{d^{3}\B{#1}}{(2\pi)^{3}}\left(-\frac{i}{2}\right)\:}}
\newcommand{\intpq}{\int{\frac{d^{3}\B{p}d^{3}\B{q}}{(2\pi)^{3}(2\pi)^{3}}\:}}
\newcommand{\intEpq}{\int{\frac{d^{3}\B{p}d^{3}\B{q}}{(2\pi)^{3}(2\pi)^{3}}\frac{1}{2E_{p}2E_{q}}\:}}
\newcommand{\inttt}[1]{\int{d^{3}\B{#1}\:}}
\newcommand{\intx}[1]{\int{d#1\:}}
\newcommand{\intiv}[1]{\int{d^{4}#1\:}}
\newcommand{\dotp}[2]{\B{#1}\cdot\B{#2}}
\newcommand{\R}{\mathbb{R}}
\newcommand{\arr}{\rightarrow}
\newcommand{\phipb}{\phi^{*}}
\newcommand{\phipf}{\phi_{*}}
\renewcommand{\b}[1]{\textbf{#1}}
\renewcommand{\i}[1]{\textit{#1}}
\newcommand{\Tc}{\mathcal{T}}
\newcommand{\Ts}[2]{\mathcal{T}_{#1}(\mathcal{#2})}
\newcommand{\Ds}[2]{\mathcal{T}^{*}_{#1}(\mathcal{#2})}
\renewcommand{\O}{\mathcal{O}}
\newcommand{\ol}[1]{\overline{#1}}
\newcommand{\ul}[1]{\underline{#1}}
\newcommand{\Dd}{\mathcal{D}}
\newcommand{\half}{\tfrac{1}{2}}
\newcommand{\bs}[1]{\boldsymbol{#1}}
\newcommand{\tl}[1]{\tilde{#1}}
\newcommand{\lorentz}{\mathbb{L}}
\newcommand{\matr}[2]{\left(\begin{array}{#1} #2\end{array}\right)}
\newcommand{\ee}{{\'e}}
\newcommand{\PP}{\mathbb{P}}
\newcommand{\EXP}[1]{\left\langle #1 \right\rangle}
\newcommand{\Tp}[1]{\tensor{\perp}{#1}}
\newcommand{\TRc}[1]{\tensor{\mathcal{R}}{#1}}
\newcommand{\TR}[1]{\tensor{R}{#1}}
\newcommand{\TK}[1]{\tensor{\kappa}{#1}}
\newcommand{\TN}[2]{\tensor{N}{_{#1}^{#2}}}
\newcommand{\TNs}[2]{\tensor{{N^{*}}}{_{\dot{#1}}^{\dot{#2}}}}
\newcommand{\dal}{\dot{\alpha}}
\newcommand{\dbe}{\dot{\beta}}
\newcommand{\ins}{\int{\frac{d\phi d\psi_{1}d\psi_{2}}{\sqrt{2\pi}}\:}}
\newcommand{\diraceq}{(i\cancel{\partial}-m)\psi}
\newcommand{\conjdirac}{\ol{\psi}(-\overleftarrow{\cancel{\partial}}-m)}
\newcommand{\cp}{\cancel{\partial}}
\newcommand{\gam}[1]{\gamma^{#1}}
\newcommand{\gamu}{\gam{\mu}}
\newcommand{\ganu}{\gam{\nu}}
\newcommand{\gamo}{\matr{cc}{0&I_{2}\\I_{2}&0}}
\newcommand{\gami}{\matr{cc}{0 &\sigma^{i}\\-\sigma{i}&0}}
\newcommand{\can}[1]{\cancel{#1}}
\newcommand{\pr}{\partial}
\newcommand{\PL}{\frac{1}{2}(1-\gam{5})}
\newcommand{\PR}{\frac{1}{2}(1+\gam{5})}
\newcommand{\psib}{\ol{\psi}}
\newcommand{\chib}{\ol{\chi}}
\newcommand{\cD}{\cancel{D}}
\newcommand{\hP}{\hat{P}}
\newcommand{\hT}{\hat{T}}
\newcommand{\poinc}{Poincar{\'e} }
\newcommand{\schro}{Schr{\"o}dinger }
\newcommand{\Pc}{\mathcal{P}}
\newcommand{\itmze}[1]{\begin{itemize} #1 \end{itemize}}
\newcommand{\tM}{\tilde{\M}}
\newcommand{\Lg}{\mathfrak{g}}
\newcommand{\Tf}[1]{\tensor{f}{#1}}
\newcommand{\PB}[2]{\left\{#1,#2\right\}_{PB}}
\newcommand{\nF}{(-)^{F}}
\newcommand{\hC}{\hat{C}}
\newcommand{\schw}{\left(1-\frac{2M}{r}\right)}
\newcommand{\schwarzchild}{ds^{2}=-\schw dt^{2}+\schw^{-1}dr^{2}+r^{2}d\Omega^{2}}
\newcommand{\Dxi}{\D{}{x^{i}}}
\newcommand{\Dxj}{\D{}{x^{j}}}
\newcommand{\dxi}{dx^{i}}
\newcommand{\dxj}{dx^{j}}
\newcommand{\Hc}{\mathcal{H}}
\newcommand{\inttsig}{\int{dt\:}\oint{d\sigma\:}}
\newcommand{\dD}[2]{\frac{\delta #1}{\delta #2}}
\newcommand{\lrarr}{\leftrightarrow}
\newcommand{\dX}{\dot{X}}
\newcommand{\Xd}{{X'}}
\newcommand{\enumi}[1]{\begin{enumerate}[(i).]#1\end{enumerate}}
\newcommand{\Uc}{\mathcal{U}}
\newcommand{\TB}[1]{\tensor{B}{#1}}
\newcommand{\lrb}[1]{\left(#1\right)}
\newcommand{\lrsq}[1]{\left[#1\right]}
\newcommand{\lrc}[1]{\left\{#1\right\}}
\newcommand{\intt}{\int{dt\:}}
\newcommand{\tal}{\tilde{\alpha}}
\newcommand{\al}{\alpha}
\newcommand{\piT}{\pi T}
\newcommand{\sqfr}[2]{\sqrt{\frac{#1}{#2}}}
\newcommand{\frsqi}[2]{\frac{\sqrt{#1}}{#2}}
\newcommand{\frsqii}[2]{\frac{#1}{\sqrt{#2}}}
\newcommand{\tL}{\tilde{L}}
\newcommand{\Half}{\frac{1}{2}}
\newcommand{\Z}{\mathbb{Z}}
\newcommand{\intpisig}{\int_{0}^{\pi}d\sigma\:}
\newcommand{\vac}{\ket{0}}
\newcommand{\phys}{\ket{\text{phys}}}
\newcommand{\hal}{\hat{\alpha}}
\newcommand{\hN}{\hat{N}}
\newcommand{\hA}{\hat{A}}
\newcommand{\TP}[1]{\tensor{P}{#1}}
\newcommand{\TBh}[1]{\tensor{\hat{B}}{#1}}
\newcommand{\hS}{\hat{S}}
\newcommand{\Tab}[1]{\tensor{#1}{^a_b}}
\newcommand{\omh}{\hat{\omega}}
\newcommand{\sigh}{\hat{\sigma}}
\newcommand{\Bh}{\hat{B}}
\newcommand{\hB}{\Bh}
\newcommand{\Braq}[1]{\Braket{q_{f}}{#1}{q_{i}}}
\newcommand{\inttx}{\intx{^3\B{x}}}
\newcommand{\pb}[1]{\{#1\}_{\text{PB}}}
\newcommand{\delal}{\delta_{\alpha}}
\newcommand{\delxi}{\delta_{\xi}}
\newcommand{\deleps}{\delta_{\epsilon}}
\newcommand{\intsig}{\oint d\sigma\:}
\newcommand{\bal}{\bs{\al}}
\newcommand{\tbal}{\tilde{\bal}}
\newcommand{\C}{\mathbb{C}}
\newcommand{\Qc}{\mathcal{Q}}
\newcommand{\TF}[1]{\tensor{F}{#1}}
\newcommand{\scri}{\mathcal{I}}
\newcommand{\scrip}{\scri^{+}}
\newcommand{\scrim}{\scri^{-}}
\newcommand{\Bc}{\mathcal{B}}
\newcommand{\Wc}{\mathcal{W}}
\newcommand{\Ac}{\mathcal{A}}
\newcommand{\Up}{U^{+}}
\newcommand{\Vp}{V^{+}}
\newcommand{\Um}{U^{-}}
\newcommand{\Vm}{V^{-}}
\newcommand{\Mg}{(\M,g)}
\newcommand{\s}{\star}
\newcommand{\Sc}{\mathcal{S}}
\newcommand{\alb}{\bar{\al}}
\newcommand{\nsg}{\vartriangleleft}
\newcommand{\idl}{\vartriangleleft}
\newcommand{\Q}{\mathbb{Q}}
\newcommand{\F}{\mathbb{F}}
\newcommand{\norm}[1]{\left|\left| #1 \right|\right|}
\newcommand{\intR}[2]{\int_{\mathbb{R}} #1 \, d#2}
\newcommand{\E}{\mathbb{E}}
\newcommand{\eqsys}[1]{\begin{subequations}\algn{#1}\end{subequations}}
\newcommand{\eqsyslab}[2]{\begin{subequations} \label{eq:#2} \algn{#1}\end{subequations}}
\newcommand{\Fc}{\mathcal{F}}
\newcommand{\EE}[1]{\mathbb{E}\lrsq{#1}}
\newcommand{\EEC}[2]{\EE{\left. #1 \right| #2}}
\newcommand{\intab}{\int_{a}^{b}}
\newcommand{\DD}[2]{\frac{D #1}{D #2}}
\newcommand{\eps}{\epsilon}
\newcommand{\spd}{symmetric positive definite }
\newcommand{\NN}{\mathbb{N}}
\newcommand{\hch}{\hat{\chi}}
\newcommand{\hps}{\hat{\psi}}
\newcommand{\be}{\beta}
\newcommand{\TJi}[1]{\tensor{(J^{-1})}{#1}}
\newcommand{\Db}{\mathbb{D}}
\newcommand{\abs}[1]{\left| #1 \right|}
\newcommand{\lesim}{\lesssim}
\newcommand{\lessim}{\lesssim}
\newcommand{\lamdba}{\lambda}
\newcommand{\indnt}{\text{  }\text{  }\text{  }\text{  }\text{  }\text{  }\text{  }\text{  }}
\newcommand{\bbk}{\bar{\bar{\kappa}}}
\newcommand{\ok}{\tfrac{1}{\kappa}}
\newcommand{\vk}{\lrb{v,\tfrac{1}{\kappa}}}
\newcommand{\vkp}{\lrb{v',\tfrac{1}{\kappa}}}




\DeclareMathOperator{\diag}{diag}
\DeclareMathOperator{\Tr}{Tr}
\DeclareMathOperator{\Div}{div}
\DeclareMathOperator{\curl}{curl}
\DeclareMathOperator{\ad}{ad}
\DeclareMathOperator{\Ad}{Ad}
\DeclareMathOperator{\Orb}{Orb}
\DeclareMathOperator{\Stab}{Stab}
\DeclareMathOperator{\sgn}{sgn}
\DeclareMathOperator{\vol}{vol}
\DeclareMathOperator{\Span}{span}
\DeclareMathOperator{\sign}{sign}
\DeclareMathOperator{\Int}{Int}
\DeclareMathOperator{\area}{area}
\DeclareMathOperator{\hcf}{hcf}
\DeclareMathOperator{\lcm}{lcm}
\DeclareMathOperator{\Syl}{Syl}
\DeclareMathOperator{\im}{Im}
\DeclareMathOperator{\Ann}{Ann}
\DeclareMathOperator{\Cov}{Cov}
\DeclareMathOperator{\mse}{mse}
\DeclareMathOperator{\Var}{Var}
\DeclareMathOperator{\Diff}{Diff}
\DeclareMathOperator{\supp}{supp}
\DeclareMathOperator{\cn}{cn}
\DeclareMathOperator{\sn}{sn}
\DeclareMathOperator{\dn}{dn}
\DeclareMathOperator{\cosech}{cosech}
\DeclareMathOperator{\sech}{sech}





\theoremstyle{definition}
\newtheorem{defn}{Definition}[section] 
\newtheorem{examp}[defn]{Example}
\theoremstyle{plain}
\newtheorem{lem}[defn]{Lemma}
\newtheorem{thm}[defn]{Theorem}
\newtheorem{cor}[defn]{Corollary}
\newtheorem{prop}[defn]{Proposition}
\newtheorem{conj}[defn]{Conjecture}

\newcommand{\Theoremproof}[2]{
\begin{thm}
#1 
\end{thm}
\begin{proof}
#2
\end{proof}
}
\newcommand{\Theoremnoproof}[1]{
\begin{thm}
#1 
\end{thm}
}
\newcommand{\Lemmaproof}[2]{
\begin{lem}
#1 
\end{lem}
\begin{proof}
#2
\end{proof}
}
\newcommand{\Lemmanoproof}[1]{
\begin{lem}
#1 
\end{lem}
}
\newcommand{\Corollaryproof}[2]{
\begin{cor}
#1 
\end{cor}
\begin{proof}
#2
\end{proof}
}
\newcommand{\Corollarynoproof}[1]{
\begin{cor}
#1 
\end{cor}
}
\newcommand{\Definition}[2]{
\begin{defn}[#1]
#2 
\end{defn}
}
\newcommand{\Propproof}[2]{
\begin{prop}
#1 
\end{prop}
\begin{proof}
#2
\end{proof}
}
\newcommand{\Propnoproof}[1]{
\begin{prop}
#1 
\end{prop}
}

\crefname{lem}{Lemma}{Lemmas}
\Crefname{lem}{Lemma}{Lemmas}
\crefname{thm}{Theorem}{Theorems}
\Crefname{thm}{Theorem}{Theorems}
\crefname{def}{Definition}{Definitions}
\Crefname{def}{Definition}{Definitions}
\crefname{examp}{Example}{Examples}
\Crefname{examp}{Example}{Examples}
\crefname{figure}{Figure}{Figures}%
\crefname{table}{Table}{Tables}



\newcommand{\makeplots}[8]{

\IfFileExists{./MESHES/#2_#3/results_std/plots/#1_allyears_#4_#5_layer_#8.png}{
\IfFileExists{./MESHES/#2_#3/results_std/plots/#1_allyears_#6_#7_layer_#8.png}{
	\begin{figure}[H]
		\centering
		\begin{subfigure}{0.45\textwidth}
  			\centering
 			 \includegraphics[width=0.9\linewidth]{./MESHES/#2_#3/results_std/plots/#1_allyears_#4_#5_layer_#8.png}
		\end{subfigure}%
		\begin{subfigure}{0.45\textwidth}
 			 \centering
 			 \includegraphics[width=0.9\linewidth]{./MESHES/#2_#3/results_std/plots/#1_allyears_#6_#7_layer_#8.png}
		\end{subfigure}
	\end{figure}
	}
	\text{ }
}



\IfFileExists{./MESHES/#2_#3/results_std/plots/#1_fluct_sp_#4_#5_layer_#8.png}{
\IfFileExists{./MESHES/#2_#3/results_std/plots/#1_fluct_sp_#6_#7_layer_#8.png}{
	\begin{figure}[H]
		\centering
		\begin{subfigure}{0.45\textwidth}
  			\centering
 			 \includegraphics[width=0.9\linewidth]{./MESHES/#2_#3/results_std/plots/#1_fluct_sp_#4_#5_layer_#8.png}
		\end{subfigure}%
		\begin{subfigure}{0.45\textwidth}
 			 \centering
 			 \includegraphics[width=0.9\linewidth]{./MESHES/#2_#3/results_std/plots/#1_fluct_sp_#6_#7_layer_#8.png}
		\end{subfigure}
	\end{figure}
}
	\text{ }
}

\IfFileExists{./MESHES/#2_#3/results_std/plots/#1_fluct_t_#4_#5_layer_#8.png}{
\IfFileExists{./MESHES/#2_#3/results_std/plots/#1_fluct_t_#6_#7_layer_#8.png}{
	\begin{figure}[H]
		\centering
		\begin{subfigure}{0.45\textwidth}
  			\centering
 			 \includegraphics[width=0.9\linewidth]{./MESHES/#2_#3/results_std/plots/#1_fluct_t_#4_#5_layer_#8.png}
		\end{subfigure}%
		\begin{subfigure}{0.45\textwidth}
 			 \centering
 			 \includegraphics[width=0.9\linewidth]{./MESHES/#2_#3/results_std/plots/#1_fluct_t_#6_#7_layer_#8.png}
		\end{subfigure}
	\end{figure}
}
	\text{ }
}


}

\newcommand{\layernotimeplot}[4]{

\IfFileExists{./MESHES/#2_#3/results_std/plots/#1_tav_layer_#4.png}{
	\begin{figure}[H]
		\centering
		 \includegraphics[width=0.45\linewidth]{./MESHES/#2_#3/results_std/plots/#1_tav_layer_#4.png}
	\end{figure}
}

}

\newcommand{\timenolayerplot}[7]{

\IfFileExists{./MESHES/#2_#3/results_std/plots/#1_strat_#4_#5.png}{
\IfFileExists{./MESHES/#2_#3/results_std/plots/#1_strat_#6_#7.png}{
	\begin{figure}[H]
		\centering
		\begin{subfigure}{0.45\textwidth}
  			\centering
 			 \includegraphics[width=0.9\linewidth]{./MESHES/#2_#3/results_std/plots/#1_strat_#4_#5.png}
		\end{subfigure}%
		\begin{subfigure}{0.45\textwidth}
 			 \centering
 			 \includegraphics[width=0.9\linewidth]{./MESHES/#2_#3/results_std/plots/#1_strat_#6_#7.png}
		\end{subfigure}
	\end{figure}
	}
		\text{ }
}

}

\newcommand{\nolayernotimeplot}[3]{

\IfFileExists{./MESHES/#2_#3/results_std/plots/#1_timeseries.png}{
	\begin{figure}[H]
		\centering
		 \includegraphics[width=0.45\linewidth]{./MESHES/#2_#3/results_std/plots/#1_timeseries.png}
	\end{figure}
}

\IfFileExists{./MESHES/#2_#3/results_std/plots/#1_tav.png}{
	\begin{figure}[H]
		\centering
		 \includegraphics[width=0.45\linewidth]{./MESHES/#2_#3/results_std/plots/#1_tav.png}
	\end{figure}
}

}


\newcommand{\makeplotsnolayers}[7]{

\IfFileExists{./MESHES/#2_#3/results_std/plots/#1_allyears_#4_#5.png}{
\IfFileExists{./MESHES/#2_#3/results_std/plots/#1_allyears_#6_#7.png}{
	\begin{figure}[H]
		\centering
		\begin{subfigure}{0.45\textwidth}
  			\centering
 			 \includegraphics[width=0.9\linewidth]{./MESHES/#2_#3/results_std/plots/#1_allyears_#4_#5.png}
		\end{subfigure}%
		\begin{subfigure}{0.45\textwidth}
 			 \centering
 			 \includegraphics[width=0.9\linewidth]{./MESHES/#2_#3/results_std/plots/#1_allyears_#6_#7.png}
		\end{subfigure}
	\end{figure}
	}
		\text{ }
}

\IfFileExists{./MESHES/#2_#3/results_std/plots/#1_fluct_sp_#4_#5.png}{
\IfFileExists{./MESHES/#2_#3/results_std/plots/#1_fluct_sp_#6_#7.png}{
	\begin{figure}[H]
		\centering
		\begin{subfigure}{0.45\textwidth}
  			\centering
 			 \includegraphics[width=0.9\linewidth]{./MESHES/#2_#3/results_std/plots/#1_fluct_sp_#4_#5.png}
		\end{subfigure}%
		\begin{subfigure}{0.45\textwidth}
 			 \centering
 			 \includegraphics[width=0.9\linewidth]{./MESHES/#2_#3/results_std/plots/#1_fluct_sp_#6_#7.png}
		\end{subfigure}
	\end{figure}
	}
		\text{ }
}

\IfFileExists{./MESHES/#2_#3/results_std/plots/#1_fluct_t_#4_#5.png}{
\IfFileExists{./MESHES/#2_#3/results_std/plots/#1_fluct_t_#6_#7.png}{
	\begin{figure}[H]
		\centering
		\begin{subfigure}{0.45\textwidth}
  			\centering
 			 \includegraphics[width=0.9\linewidth]{./MESHES/#2_#3/results_std/plots/#1_fluct_t_#4_#5.png}
		\end{subfigure}%
		\begin{subfigure}{0.45\textwidth}
 			 \centering
 			 \includegraphics[width=0.9\linewidth]{./MESHES/#2_#3/results_std/plots/#1_fluct_t_#6_#7.png}
		\end{subfigure}
	\end{figure}
	}
		\text{ }
}

}

\newcommand{\allplot}[9]{

\makeplots{unod}{#1}{#2}{#3}{#4}{#5}{#6}{#7}
\makeplots{unod}{#1}{#2}{#3}{#4}{#5}{#6}{#8}
\makeplots{unod}{#1}{#2}{#3}{#4}{#5}{#6}{#9}
\layernotimeplot{unod}{#1}{#2}{#7}
\layernotimeplot{unod}{#1}{#2}{#8}
\layernotimeplot{unod}{#1}{#2}{#9}
\timenolayerplot{unod}{#1}{#2}{#3}{#4}{#5}{#6}
\nolayernotimeplot{unod}{#1}{#2}

\newpage
\makeplots{vnod}{#1}{#2}{#3}{#4}{#5}{#6}{#7}
\makeplots{vnod}{#1}{#2}{#3}{#4}{#5}{#6}{#8}
\makeplots{vnod}{#1}{#2}{#3}{#4}{#5}{#6}{#9}
\layernotimeplot{vnod}{#1}{#2}{#7}
\layernotimeplot{vnod}{#1}{#2}{#8}
\layernotimeplot{vnod}{#1}{#2}{#9}
\timenolayerplot{vnod}{#1}{#2}{#3}{#4}{#5}{#6}
\nolayernotimeplot{vnod}{#1}{#2}

\newpage
\makeplots{w}{#1}{#2}{#3}{#4}{#5}{#6}{#7}
\makeplots{w}{#1}{#2}{#3}{#4}{#5}{#6}{#8}
\makeplots{w}{#1}{#2}{#3}{#4}{#5}{#6}{#9}
\layernotimeplot{w}{#1}{#2}{#7}
\layernotimeplot{w}{#1}{#2}{#8}
\layernotimeplot{w}{#1}{#2}{#9}
\timenolayerplot{w}{#1}{#2}{#3}{#4}{#5}{#6}
\nolayernotimeplot{w}{#1}{#2}

\newpage
\makeplots{temp}{#1}{#2}{#3}{#4}{#5}{#6}{#7}
\makeplots{temp}{#1}{#2}{#3}{#4}{#5}{#6}{#8}
\makeplots{temp}{#1}{#2}{#3}{#4}{#5}{#6}{#9}
\layernotimeplot{temp}{#1}{#2}{#7}
\layernotimeplot{temp}{#1}{#2}{#8}
\layernotimeplot{temp}{#1}{#2}{#9}
\timenolayerplot{temp}{#1}{#2}{#3}{#4}{#5}{#6}
\nolayernotimeplot{temp}{#1}{#2}

\newpage
\makeplots{rhof}{#1}{#2}{#3}{#4}{#5}{#6}{#7}
\makeplots{rhof}{#1}{#2}{#3}{#4}{#5}{#6}{#8}
\makeplots{rhof}{#1}{#2}{#3}{#4}{#5}{#6}{#9}
\layernotimeplot{rhof}{#1}{#2}{#7}
\layernotimeplot{rhof}{#1}{#2}{#8}
\layernotimeplot{rhof}{#1}{#2}{#9}
\timenolayerplot{rhof}{#1}{#2}{#3}{#4}{#5}{#6}
\nolayernotimeplot{rhofmid}{#1}{#2}

\newpage
\makeplots{ke}{#1}{#2}{#3}{#4}{#5}{#6}{#7}
\makeplots{ke}{#1}{#2}{#3}{#4}{#5}{#6}{#8}
\makeplots{ke}{#1}{#2}{#3}{#4}{#5}{#6}{#9}
\layernotimeplot{ke}{#1}{#2}{#7}
\layernotimeplot{ke}{#1}{#2}{#8}
\layernotimeplot{ke}{#1}{#2}{#9}
\timenolayerplot{ke}{#1}{#2}{#3}{#4}{#5}{#6}
\nolayernotimeplot{ke}{#1}{#2}

\newpage
\makeplotsnolayers{ssh}{#1}{#2}{#3}{#4}{#5}{#6}
\timenolayerplot{ssh}{#1}{#2}{#3}{#4}{#5}{#6}
\nolayernotimeplot{ssh}{#1}{#2}

}

\newcommand{\momplot}[3]{

\IfFileExists{./DATA/513/#1_#2_#3.png}{
	\begin{figure}[H]
		\centering
		 \includegraphics[width=\linewidth]{./DATA/513/#1_#2_#3.png}
	\end{figure}
}

}

\newcommand{\momplottav}[3]{

\IfFileExists{./DATA/513/#1_tav_#2_#3.png}{
	\begin{figure}[H]
		\centering
		 \includegraphics[width=\linewidth]{./DATA/513/#1_tav_#2_#3.png}
	\end{figure}
}

}

\newcommand{\mitgcmplot}[3]{

\IfFileExists{../../../../../Documents/MITgcm/verification/tutorial_baroclinic_gyre/results/plots/#1_#2_#3.png}{
	\begin{figure}[H]
		\centering
		 \includegraphics[width=0.7\linewidth]{../../../../../Documents/MITgcm/verification/tutorial_baroclinic_gyre/results/plots/#1_#2_#3.png}
	\end{figure}
}

}
