\documentclass[10pt]{article}

\usepackage{amsfonts}
\usepackage{amsthm}
\usepackage[mathscr]{euscript}
\usepackage{amsmath}
\usepackage{amssymb}
\usepackage{tensor}
\usepackage{fullpage}
\usepackage{enumerate}
\usepackage{graphicx,caption}
\usepackage{float}
\usepackage{url}
\usepackage[hidelinks]{hyperref}
\usepackage{cleveref}
\usepackage[font=footnotesize,labelfont=footnotesize]{caption}
\usepackage[ansinew]{inputenc}
\usepackage{pbox}
\usepackage{cancel}
\usepackage{accents}
\usepackage{subcaption}
\usepackage{color}


\setlength\parindent{0pt}
\linespread{1.2}

\newcommand{\D}[2]{\frac{d #1}{d #2}}
\newcommand{\PD}[2]{\frac{\partial #1}{\partial #2}}
\newcommand{\eq}[1]{\begin{equation} #1 \end{equation}}
\newcommand{\eqlab}[2]{\begin{equation} #1 \label{eq:#2} \end{equation}}
\newcommand{\eqnn}[1]{\eq{#1 \nonumber}}
\newcommand{\algn}[1]{\begin{align} #1 \end{align}}
\newcommand{\algnn}[1]{\begin{align*} #1 \end{align*}}
\newcommand{\bra}[1]{\left\langle #1 \right|}
\newcommand{\ket}[1]{\left| #1 \right\rangle}
\newcommand{\braket}[2]{\langle #1 | #2 \rangle}
\newcommand{\Braket}[3]{\langle #1 | #2 | #3 \rangle}
\newcommand{\B}[1]{\mathbf{#1}}
\newcommand{\tvec}[2]{\left(\begin{array}{c} #1 \\ #2 \end{array}\right)}
\newcommand{\tvecs}[2]{\left(\begin{smallmatrix} #1\\#2\end{smallmatrix}\right)}
\newcommand{\twobytwomatrix}[4]{\left(\begin{array}{cc} #1 & #2 \\ #3 & #4\end{array}\right)}
\newcommand{\threevec}[3]{\left(\begin{array}{c} #1 \\ #2 \\ #3 \end{array}\right)}
\newcommand{\ph}{\varphi}
\newcommand{\TL}[1]{\tensor{\Lambda}{#1}}
\newcommand{\TG}[1]{\tensor{\Gamma}{#1}}
\newcommand{\TT}[1]{\tensor{T}{#1}}
\newcommand{\TA}[1]{\tensor{A}{#1}}
\newcommand{\Td}[1]{\tensor{\delta}{#1}}
\newcommand{\TAinv}[1]{\tensor{{(A^{-1})}}{#1}}
\newcommand{\M}{\mathcal{M}}
\newcommand{\N}{\mathcal{N}}
\newcommand{\Dc}{\mathcal{D}}
\renewcommand{\L}{\mathcal{L}}
\newcommand{\intp}[1]{\int{\frac{d^{3}\B{#1}}{(2\pi)^{3}}\:}}
\newcommand{\intE}[1]{\int{\frac{d^{3}\B{#1}}{(2\pi)^{3}}\frac{1}{2E_{#1}}\:}}
\newcommand{\intpi}[1]{\int{\frac{d^{3}\B{#1}}{(2\pi)^{3}}\left(-\frac{i}{2}\right)\:}}
\newcommand{\intpq}{\int{\frac{d^{3}\B{p}d^{3}\B{q}}{(2\pi)^{3}(2\pi)^{3}}\:}}
\newcommand{\intEpq}{\int{\frac{d^{3}\B{p}d^{3}\B{q}}{(2\pi)^{3}(2\pi)^{3}}\frac{1}{2E_{p}2E_{q}}\:}}
\newcommand{\inttt}[1]{\int{d^{3}\B{#1}\:}}
\newcommand{\intx}[1]{\int{d#1\:}}
\newcommand{\intiv}[1]{\int{d^{4}#1\:}}
\newcommand{\dotp}[2]{\B{#1}\cdot\B{#2}}
\newcommand{\R}{\mathbb{R}}
\newcommand{\arr}{\rightarrow}
\newcommand{\phipb}{\phi^{*}}
\newcommand{\phipf}{\phi_{*}}
\renewcommand{\b}[1]{\textbf{#1}}
\renewcommand{\i}[1]{\textit{#1}}
\newcommand{\Tc}{\mathcal{T}}
\newcommand{\Ts}[2]{\mathcal{T}_{#1}(\mathcal{#2})}
\newcommand{\Ds}[2]{\mathcal{T}^{*}_{#1}(\mathcal{#2})}
\renewcommand{\O}{\mathcal{O}}
\newcommand{\ol}[1]{\overline{#1}}
\newcommand{\ul}[1]{\underline{#1}}
\newcommand{\Dd}{\mathcal{D}}
\newcommand{\half}{\tfrac{1}{2}}
\newcommand{\bs}[1]{\boldsymbol{#1}}
\newcommand{\tl}[1]{\tilde{#1}}
\newcommand{\lorentz}{\mathbb{L}}
\newcommand{\matr}[2]{\left(\begin{array}{#1} #2\end{array}\right)}
\newcommand{\ee}{{\'e}}
\newcommand{\PP}{\mathbb{P}}
\newcommand{\EXP}[1]{\left\langle #1 \right\rangle}
\newcommand{\Tp}[1]{\tensor{\perp}{#1}}
\newcommand{\TRc}[1]{\tensor{\mathcal{R}}{#1}}
\newcommand{\TR}[1]{\tensor{R}{#1}}
\newcommand{\TK}[1]{\tensor{\kappa}{#1}}
\newcommand{\TN}[2]{\tensor{N}{_{#1}^{#2}}}
\newcommand{\TNs}[2]{\tensor{{N^{*}}}{_{\dot{#1}}^{\dot{#2}}}}
\newcommand{\dal}{\dot{\alpha}}
\newcommand{\dbe}{\dot{\beta}}
\newcommand{\ins}{\int{\frac{d\phi d\psi_{1}d\psi_{2}}{\sqrt{2\pi}}\:}}
\newcommand{\diraceq}{(i\cancel{\partial}-m)\psi}
\newcommand{\conjdirac}{\ol{\psi}(-\overleftarrow{\cancel{\partial}}-m)}
\newcommand{\cp}{\cancel{\partial}}
\newcommand{\gam}[1]{\gamma^{#1}}
\newcommand{\gamu}{\gam{\mu}}
\newcommand{\ganu}{\gam{\nu}}
\newcommand{\gamo}{\matr{cc}{0&I_{2}\\I_{2}&0}}
\newcommand{\gami}{\matr{cc}{0 &\sigma^{i}\\-\sigma{i}&0}}
\newcommand{\can}[1]{\cancel{#1}}
\newcommand{\pr}{\partial}
\newcommand{\PL}{\frac{1}{2}(1-\gam{5})}
\newcommand{\PR}{\frac{1}{2}(1+\gam{5})}
\newcommand{\psib}{\ol{\psi}}
\newcommand{\chib}{\ol{\chi}}
\newcommand{\cD}{\cancel{D}}
\newcommand{\hP}{\hat{P}}
\newcommand{\hT}{\hat{T}}
\newcommand{\poinc}{Poincar{\'e} }
\newcommand{\schro}{Schr{\"o}dinger }
\newcommand{\Pc}{\mathcal{P}}
\newcommand{\itmze}[1]{\begin{itemize} #1 \end{itemize}}
\newcommand{\tM}{\tilde{\M}}
\newcommand{\Lg}{\mathfrak{g}}
\newcommand{\Tf}[1]{\tensor{f}{#1}}
\newcommand{\PB}[2]{\left\{#1,#2\right\}_{PB}}
\newcommand{\nF}{(-)^{F}}
\newcommand{\hC}{\hat{C}}
\newcommand{\schw}{\left(1-\frac{2M}{r}\right)}
\newcommand{\schwarzchild}{ds^{2}=-\schw dt^{2}+\schw^{-1}dr^{2}+r^{2}d\Omega^{2}}
\newcommand{\Dxi}{\D{}{x^{i}}}
\newcommand{\Dxj}{\D{}{x^{j}}}
\newcommand{\dxi}{dx^{i}}
\newcommand{\dxj}{dx^{j}}
\newcommand{\Hc}{\mathcal{H}}
\newcommand{\inttsig}{\int{dt\:}\oint{d\sigma\:}}
\newcommand{\dD}[2]{\frac{\delta #1}{\delta #2}}
\newcommand{\lrarr}{\leftrightarrow}
\newcommand{\dX}{\dot{X}}
\newcommand{\Xd}{{X'}}
\newcommand{\enumi}[1]{\begin{enumerate}[(i).]#1\end{enumerate}}
\newcommand{\Uc}{\mathcal{U}}
\newcommand{\TB}[1]{\tensor{B}{#1}}
\newcommand{\lrb}[1]{\left(#1\right)}
\newcommand{\lrsq}[1]{\left[#1\right]}
\newcommand{\lrc}[1]{\left\{#1\right\}}
\newcommand{\intt}{\int{dt\:}}
\newcommand{\tal}{\tilde{\alpha}}
\newcommand{\al}{\alpha}
\newcommand{\piT}{\pi T}
\newcommand{\sqfr}[2]{\sqrt{\frac{#1}{#2}}}
\newcommand{\frsqi}[2]{\frac{\sqrt{#1}}{#2}}
\newcommand{\frsqii}[2]{\frac{#1}{\sqrt{#2}}}
\newcommand{\tL}{\tilde{L}}
\newcommand{\Half}{\frac{1}{2}}
\newcommand{\Z}{\mathbb{Z}}
\newcommand{\intpisig}{\int_{0}^{\pi}d\sigma\:}
\newcommand{\vac}{\ket{0}}
\newcommand{\phys}{\ket{\text{phys}}}
\newcommand{\hal}{\hat{\alpha}}
\newcommand{\hN}{\hat{N}}
\newcommand{\hA}{\hat{A}}
\newcommand{\TP}[1]{\tensor{P}{#1}}
\newcommand{\TBh}[1]{\tensor{\hat{B}}{#1}}
\newcommand{\hS}{\hat{S}}
\newcommand{\Tab}[1]{\tensor{#1}{^a_b}}
\newcommand{\omh}{\hat{\omega}}
\newcommand{\sigh}{\hat{\sigma}}
\newcommand{\Bh}{\hat{B}}
\newcommand{\hB}{\Bh}
\newcommand{\Braq}[1]{\Braket{q_{f}}{#1}{q_{i}}}
\newcommand{\inttx}{\intx{^3\B{x}}}
\newcommand{\pb}[1]{\{#1\}_{\text{PB}}}
\newcommand{\delal}{\delta_{\alpha}}
\newcommand{\delxi}{\delta_{\xi}}
\newcommand{\deleps}{\delta_{\epsilon}}
\newcommand{\intsig}{\oint d\sigma\:}
\newcommand{\bal}{\bs{\al}}
\newcommand{\tbal}{\tilde{\bal}}
\newcommand{\C}{\mathbb{C}}
\newcommand{\Qc}{\mathcal{Q}}
\newcommand{\TF}[1]{\tensor{F}{#1}}
\newcommand{\scri}{\mathcal{I}}
\newcommand{\scrip}{\scri^{+}}
\newcommand{\scrim}{\scri^{-}}
\newcommand{\Bc}{\mathcal{B}}
\newcommand{\Wc}{\mathcal{W}}
\newcommand{\Ac}{\mathcal{A}}
\newcommand{\Up}{U^{+}}
\newcommand{\Vp}{V^{+}}
\newcommand{\Um}{U^{-}}
\newcommand{\Vm}{V^{-}}
\newcommand{\Mg}{(\M,g)}
\newcommand{\s}{\star}
\newcommand{\Sc}{\mathcal{S}}
\newcommand{\alb}{\bar{\al}}
\newcommand{\nsg}{\vartriangleleft}
\newcommand{\idl}{\vartriangleleft}
\newcommand{\Q}{\mathbb{Q}}
\newcommand{\F}{\mathbb{F}}
\newcommand{\norm}[1]{\left|\left| #1 \right|\right|}
\newcommand{\intR}[2]{\int_{\mathbb{R}} #1 \, d#2}
\newcommand{\E}{\mathbb{E}}
\newcommand{\eqsys}[1]{\begin{subequations}\algn{#1}\end{subequations}}
\newcommand{\eqsyslab}[2]{\begin{subequations} \label{eq:#2} \algn{#1}\end{subequations}}
\newcommand{\Fc}{\mathcal{F}}
\newcommand{\EE}[1]{\mathbb{E}\lrsq{#1}}
\newcommand{\EEC}[2]{\EE{\left. #1 \right| #2}}
\newcommand{\intab}{\int_{a}^{b}}
\newcommand{\DD}[2]{\frac{D #1}{D #2}}
\newcommand{\eps}{\epsilon}
\newcommand{\spd}{symmetric positive definite }
\newcommand{\NN}{\mathbb{N}}
\newcommand{\hch}{\hat{\chi}}
\newcommand{\hps}{\hat{\psi}}
\newcommand{\be}{\beta}
\newcommand{\TJi}[1]{\tensor{(J^{-1})}{#1}}
\newcommand{\Db}{\mathbb{D}}
\newcommand{\abs}[1]{\left| #1 \right|}
\newcommand{\lesim}{\lesssim}
\newcommand{\lessim}{\lesssim}
\newcommand{\lamdba}{\lambda}
\newcommand{\indnt}{\text{  }\text{  }\text{  }\text{  }\text{  }\text{  }\text{  }\text{  }}
\newcommand{\bbk}{\bar{\bar{\kappa}}}
\newcommand{\ok}{\tfrac{1}{\kappa}}
\newcommand{\vk}{\lrb{v,\tfrac{1}{\kappa}}}
\newcommand{\vkp}{\lrb{v',\tfrac{1}{\kappa}}}




\DeclareMathOperator{\diag}{diag}
\DeclareMathOperator{\Tr}{Tr}
\DeclareMathOperator{\Div}{div}
\DeclareMathOperator{\curl}{curl}
\DeclareMathOperator{\ad}{ad}
\DeclareMathOperator{\Ad}{Ad}
\DeclareMathOperator{\Orb}{Orb}
\DeclareMathOperator{\Stab}{Stab}
\DeclareMathOperator{\sgn}{sgn}
\DeclareMathOperator{\vol}{vol}
\DeclareMathOperator{\Span}{span}
\DeclareMathOperator{\sign}{sign}
\DeclareMathOperator{\Int}{Int}
\DeclareMathOperator{\area}{area}
\DeclareMathOperator{\hcf}{hcf}
\DeclareMathOperator{\lcm}{lcm}
\DeclareMathOperator{\Syl}{Syl}
\DeclareMathOperator{\im}{Im}
\DeclareMathOperator{\Ann}{Ann}
\DeclareMathOperator{\Cov}{Cov}
\DeclareMathOperator{\mse}{mse}
\DeclareMathOperator{\Var}{Var}
\DeclareMathOperator{\Diff}{Diff}
\DeclareMathOperator{\supp}{supp}
\DeclareMathOperator{\cn}{cn}
\DeclareMathOperator{\sn}{sn}
\DeclareMathOperator{\dn}{dn}
\DeclareMathOperator{\cosech}{cosech}
\DeclareMathOperator{\sech}{sech}





\theoremstyle{definition}
\newtheorem{defn}{Definition}[section] 
\newtheorem{examp}[defn]{Example}
\theoremstyle{plain}
\newtheorem{lem}[defn]{Lemma}
\newtheorem{thm}[defn]{Theorem}
\newtheorem{cor}[defn]{Corollary}
\newtheorem{prop}[defn]{Proposition}
\newtheorem{conj}[defn]{Conjecture}

\newcommand{\Theoremproof}[2]{
\begin{thm}
#1 
\end{thm}
\begin{proof}
#2
\end{proof}
}
\newcommand{\Theoremnoproof}[1]{
\begin{thm}
#1 
\end{thm}
}
\newcommand{\Lemmaproof}[2]{
\begin{lem}
#1 
\end{lem}
\begin{proof}
#2
\end{proof}
}
\newcommand{\Lemmanoproof}[1]{
\begin{lem}
#1 
\end{lem}
}
\newcommand{\Corollaryproof}[2]{
\begin{cor}
#1 
\end{cor}
\begin{proof}
#2
\end{proof}
}
\newcommand{\Corollarynoproof}[1]{
\begin{cor}
#1 
\end{cor}
}
\newcommand{\Definition}[2]{
\begin{defn}[#1]
#2 
\end{defn}
}
\newcommand{\Propproof}[2]{
\begin{prop}
#1 
\end{prop}
\begin{proof}
#2
\end{proof}
}
\newcommand{\Propnoproof}[1]{
\begin{prop}
#1 
\end{prop}
}

\crefname{lem}{Lemma}{Lemmas}
\Crefname{lem}{Lemma}{Lemmas}
\crefname{thm}{Theorem}{Theorems}
\Crefname{thm}{Theorem}{Theorems}
\crefname{def}{Definition}{Definitions}
\Crefname{def}{Definition}{Definitions}
\crefname{examp}{Example}{Examples}
\Crefname{examp}{Example}{Examples}
\crefname{figure}{Figure}{Figures}%
\crefname{table}{Table}{Tables}



\newcommand{\makeplots}[8]{

\IfFileExists{./MESHES/#2_#3/results_std/plots/#1_allyears_#4_#5_layer_#8.png}{
\IfFileExists{./MESHES/#2_#3/results_std/plots/#1_allyears_#6_#7_layer_#8.png}{
	\begin{figure}[H]
		\centering
		\begin{subfigure}{0.45\textwidth}
  			\centering
 			 \includegraphics[width=0.9\linewidth]{./MESHES/#2_#3/results_std/plots/#1_allyears_#4_#5_layer_#8.png}
		\end{subfigure}%
		\begin{subfigure}{0.45\textwidth}
 			 \centering
 			 \includegraphics[width=0.9\linewidth]{./MESHES/#2_#3/results_std/plots/#1_allyears_#6_#7_layer_#8.png}
		\end{subfigure}
	\end{figure}
	}
	\text{ }
}



\IfFileExists{./MESHES/#2_#3/results_std/plots/#1_fluct_sp_#4_#5_layer_#8.png}{
\IfFileExists{./MESHES/#2_#3/results_std/plots/#1_fluct_sp_#6_#7_layer_#8.png}{
	\begin{figure}[H]
		\centering
		\begin{subfigure}{0.45\textwidth}
  			\centering
 			 \includegraphics[width=0.9\linewidth]{./MESHES/#2_#3/results_std/plots/#1_fluct_sp_#4_#5_layer_#8.png}
		\end{subfigure}%
		\begin{subfigure}{0.45\textwidth}
 			 \centering
 			 \includegraphics[width=0.9\linewidth]{./MESHES/#2_#3/results_std/plots/#1_fluct_sp_#6_#7_layer_#8.png}
		\end{subfigure}
	\end{figure}
}
	\text{ }
}

\IfFileExists{./MESHES/#2_#3/results_std/plots/#1_fluct_t_#4_#5_layer_#8.png}{
\IfFileExists{./MESHES/#2_#3/results_std/plots/#1_fluct_t_#6_#7_layer_#8.png}{
	\begin{figure}[H]
		\centering
		\begin{subfigure}{0.45\textwidth}
  			\centering
 			 \includegraphics[width=0.9\linewidth]{./MESHES/#2_#3/results_std/plots/#1_fluct_t_#4_#5_layer_#8.png}
		\end{subfigure}%
		\begin{subfigure}{0.45\textwidth}
 			 \centering
 			 \includegraphics[width=0.9\linewidth]{./MESHES/#2_#3/results_std/plots/#1_fluct_t_#6_#7_layer_#8.png}
		\end{subfigure}
	\end{figure}
}
	\text{ }
}


}

\newcommand{\layernotimeplot}[4]{

\IfFileExists{./MESHES/#2_#3/results_std/plots/#1_tav_layer_#4.png}{
	\begin{figure}[H]
		\centering
		 \includegraphics[width=0.45\linewidth]{./MESHES/#2_#3/results_std/plots/#1_tav_layer_#4.png}
	\end{figure}
}

}

\newcommand{\timenolayerplot}[7]{

\IfFileExists{./MESHES/#2_#3/results_std/plots/#1_strat_#4_#5.png}{
\IfFileExists{./MESHES/#2_#3/results_std/plots/#1_strat_#6_#7.png}{
	\begin{figure}[H]
		\centering
		\begin{subfigure}{0.45\textwidth}
  			\centering
 			 \includegraphics[width=0.9\linewidth]{./MESHES/#2_#3/results_std/plots/#1_strat_#4_#5.png}
		\end{subfigure}%
		\begin{subfigure}{0.45\textwidth}
 			 \centering
 			 \includegraphics[width=0.9\linewidth]{./MESHES/#2_#3/results_std/plots/#1_strat_#6_#7.png}
		\end{subfigure}
	\end{figure}
	}
		\text{ }
}

}

\newcommand{\nolayernotimeplot}[3]{

\IfFileExists{./MESHES/#2_#3/results_std/plots/#1_timeseries.png}{
	\begin{figure}[H]
		\centering
		 \includegraphics[width=0.45\linewidth]{./MESHES/#2_#3/results_std/plots/#1_timeseries.png}
	\end{figure}
}

\IfFileExists{./MESHES/#2_#3/results_std/plots/#1_tav.png}{
	\begin{figure}[H]
		\centering
		 \includegraphics[width=0.45\linewidth]{./MESHES/#2_#3/results_std/plots/#1_tav.png}
	\end{figure}
}

}


\newcommand{\makeplotsnolayers}[7]{

\IfFileExists{./MESHES/#2_#3/results_std/plots/#1_allyears_#4_#5.png}{
\IfFileExists{./MESHES/#2_#3/results_std/plots/#1_allyears_#6_#7.png}{
	\begin{figure}[H]
		\centering
		\begin{subfigure}{0.45\textwidth}
  			\centering
 			 \includegraphics[width=0.9\linewidth]{./MESHES/#2_#3/results_std/plots/#1_allyears_#4_#5.png}
		\end{subfigure}%
		\begin{subfigure}{0.45\textwidth}
 			 \centering
 			 \includegraphics[width=0.9\linewidth]{./MESHES/#2_#3/results_std/plots/#1_allyears_#6_#7.png}
		\end{subfigure}
	\end{figure}
	}
		\text{ }
}

\IfFileExists{./MESHES/#2_#3/results_std/plots/#1_fluct_sp_#4_#5.png}{
\IfFileExists{./MESHES/#2_#3/results_std/plots/#1_fluct_sp_#6_#7.png}{
	\begin{figure}[H]
		\centering
		\begin{subfigure}{0.45\textwidth}
  			\centering
 			 \includegraphics[width=0.9\linewidth]{./MESHES/#2_#3/results_std/plots/#1_fluct_sp_#4_#5.png}
		\end{subfigure}%
		\begin{subfigure}{0.45\textwidth}
 			 \centering
 			 \includegraphics[width=0.9\linewidth]{./MESHES/#2_#3/results_std/plots/#1_fluct_sp_#6_#7.png}
		\end{subfigure}
	\end{figure}
	}
		\text{ }
}

\IfFileExists{./MESHES/#2_#3/results_std/plots/#1_fluct_t_#4_#5.png}{
\IfFileExists{./MESHES/#2_#3/results_std/plots/#1_fluct_t_#6_#7.png}{
	\begin{figure}[H]
		\centering
		\begin{subfigure}{0.45\textwidth}
  			\centering
 			 \includegraphics[width=0.9\linewidth]{./MESHES/#2_#3/results_std/plots/#1_fluct_t_#4_#5.png}
		\end{subfigure}%
		\begin{subfigure}{0.45\textwidth}
 			 \centering
 			 \includegraphics[width=0.9\linewidth]{./MESHES/#2_#3/results_std/plots/#1_fluct_t_#6_#7.png}
		\end{subfigure}
	\end{figure}
	}
		\text{ }
}

}

\newcommand{\allplot}[9]{

\makeplots{unod}{#1}{#2}{#3}{#4}{#5}{#6}{#7}
\makeplots{unod}{#1}{#2}{#3}{#4}{#5}{#6}{#8}
\makeplots{unod}{#1}{#2}{#3}{#4}{#5}{#6}{#9}
\layernotimeplot{unod}{#1}{#2}{#7}
\layernotimeplot{unod}{#1}{#2}{#8}
\layernotimeplot{unod}{#1}{#2}{#9}
\timenolayerplot{unod}{#1}{#2}{#3}{#4}{#5}{#6}
\nolayernotimeplot{unod}{#1}{#2}

\newpage
\makeplots{vnod}{#1}{#2}{#3}{#4}{#5}{#6}{#7}
\makeplots{vnod}{#1}{#2}{#3}{#4}{#5}{#6}{#8}
\makeplots{vnod}{#1}{#2}{#3}{#4}{#5}{#6}{#9}
\layernotimeplot{vnod}{#1}{#2}{#7}
\layernotimeplot{vnod}{#1}{#2}{#8}
\layernotimeplot{vnod}{#1}{#2}{#9}
\timenolayerplot{vnod}{#1}{#2}{#3}{#4}{#5}{#6}
\nolayernotimeplot{vnod}{#1}{#2}

\newpage
\makeplots{w}{#1}{#2}{#3}{#4}{#5}{#6}{#7}
\makeplots{w}{#1}{#2}{#3}{#4}{#5}{#6}{#8}
\makeplots{w}{#1}{#2}{#3}{#4}{#5}{#6}{#9}
\layernotimeplot{w}{#1}{#2}{#7}
\layernotimeplot{w}{#1}{#2}{#8}
\layernotimeplot{w}{#1}{#2}{#9}
\timenolayerplot{w}{#1}{#2}{#3}{#4}{#5}{#6}
\nolayernotimeplot{w}{#1}{#2}

\newpage
\makeplots{temp}{#1}{#2}{#3}{#4}{#5}{#6}{#7}
\makeplots{temp}{#1}{#2}{#3}{#4}{#5}{#6}{#8}
\makeplots{temp}{#1}{#2}{#3}{#4}{#5}{#6}{#9}
\layernotimeplot{temp}{#1}{#2}{#7}
\layernotimeplot{temp}{#1}{#2}{#8}
\layernotimeplot{temp}{#1}{#2}{#9}
\timenolayerplot{temp}{#1}{#2}{#3}{#4}{#5}{#6}
\nolayernotimeplot{temp}{#1}{#2}

\newpage
\makeplots{rhof}{#1}{#2}{#3}{#4}{#5}{#6}{#7}
\makeplots{rhof}{#1}{#2}{#3}{#4}{#5}{#6}{#8}
\makeplots{rhof}{#1}{#2}{#3}{#4}{#5}{#6}{#9}
\layernotimeplot{rhof}{#1}{#2}{#7}
\layernotimeplot{rhof}{#1}{#2}{#8}
\layernotimeplot{rhof}{#1}{#2}{#9}
\timenolayerplot{rhof}{#1}{#2}{#3}{#4}{#5}{#6}
\nolayernotimeplot{rhofmid}{#1}{#2}

\newpage
\makeplots{ke}{#1}{#2}{#3}{#4}{#5}{#6}{#7}
\makeplots{ke}{#1}{#2}{#3}{#4}{#5}{#6}{#8}
\makeplots{ke}{#1}{#2}{#3}{#4}{#5}{#6}{#9}
\layernotimeplot{ke}{#1}{#2}{#7}
\layernotimeplot{ke}{#1}{#2}{#8}
\layernotimeplot{ke}{#1}{#2}{#9}
\timenolayerplot{ke}{#1}{#2}{#3}{#4}{#5}{#6}
\nolayernotimeplot{ke}{#1}{#2}

\newpage
\makeplotsnolayers{ssh}{#1}{#2}{#3}{#4}{#5}{#6}
\timenolayerplot{ssh}{#1}{#2}{#3}{#4}{#5}{#6}
\nolayernotimeplot{ssh}{#1}{#2}

}

\newcommand{\momplot}[3]{

\IfFileExists{./DATA/513/#1_#2_#3.png}{
	\begin{figure}[H]
		\centering
		 \includegraphics[width=\linewidth]{./DATA/513/#1_#2_#3.png}
	\end{figure}
}

}

\newcommand{\momplottav}[3]{

\IfFileExists{./DATA/513/#1_tav_#2_#3.png}{
	\begin{figure}[H]
		\centering
		 \includegraphics[width=\linewidth]{./DATA/513/#1_tav_#2_#3.png}
	\end{figure}
}

}

\newcommand{\mitgcmplot}[3]{

\IfFileExists{../../../../../Documents/MITgcm/verification/tutorial_baroclinic_gyre/results/plots/#1_#2_#3.png}{
	\begin{figure}[H]
		\centering
		 \includegraphics[width=0.7\linewidth]{../../../../../Documents/MITgcm/verification/tutorial_baroclinic_gyre/results/plots/#1_#2_#3.png}
	\end{figure}
}

}


\begin{document}
\title{FESOM2}
\date{}
\maketitle
\section{Derivation of model (Danilov et al 2016)}
In what follows bold font will be used to indicate two-dimensional vectors; ordinary font may be used for vectors of three dimensions, or scalars.\\
A fluid is moving inside a domain $\M(t)$, divided into $N$ discrete and disjoint regions $V_{k}(t)$ which evolve with time:
\eqnn{
	\M(t) = \bigcup_{k=1}^{N} V_{k}(t)
} 
Each region $V_{k}$ consists of a vertically-oriented triangular prism, for which the top and bottom triangle may change in time in both height and orientation. \\
Consider the fluid contained within one of these regions. The fluid obeys Newton's second law
\algnn{
	&\D{}{t} \int_{V_{k}(t)} \rho(x,t) u(x,t) d^{3}x + \int_{\pr V_{k}(t)} \rho(x,t)u(x,t) (u(x,t) - v(x,t))\cdot n dS \\
	&\qquad  = -\int_{V_{k}(t)} \lrsq{\rho(x,t)g e_{3} + \rho f e_{3}\times u(x,t) }d^{3}x   - \int_{\pr V_{k}(t)} p(x,t) n dS
}
	where $\rho$ is the fluid density, $u = (\B{u},w)$ the three dimensional velocity and $v = (0,0,\hat{w})$ the velocity of the mesh. Now let us split the integral into its horizontal and vertical components, and denote with an over-bar the height-average of a quantity: $\bar{\cdot} = \frac{1}{h} \int_{z_{-}}^{z_{+}} \cdot$, where $z_{\pm}(\B{x})$ are the top and bottom heights of the prism. This results in the following:
	\algnn{
	&\D{}{t} \int_{A} h \ol{\rho u} dS + \int_{\pr A} h \ol{\rho u  \B{u}}\cdot n dS + \lrb{\int_{A_{+}} - \int_{A_{-}}} \rho u \tl{w} dS  \\
	& \qquad  = -\int_{A}h\ol{ \lrsq{\rho g e_{3} + \rho f e_{3}\times u} }dS - \int_{\pr A} h\ol{p} n dS - \lrb{\int_{A_{+}} - \int_{A_{-}}} p n dS
}
where $A_{\pm}$ are the end (triangular) faces of the prism, and $\tl{w} = (u -v)\cdot n$ with $n$ the normal to $A_{\pm}$. Effectively $\tl{w}$ is the flux across the ends of the prism. $A$ is the horizontal area of the prism; that is, the area of $A_{\pm}$ when they lie flat.  \\
\linebreak
Note that, for example:
\eqnn{
	\int_{A_{+}} p n dS = \int_{A} p(x,y,z_{+}(x,y)) (-\nabla z_{+} , 1) dx dy
	}
We now make the Boussinesq and hydrostatic approximations (see for example, Vallis page 71):
\algn{
	\rho & = \rho_{0} + \rho'(x,y,z,t), \quad \rho'\ll \rho_{0} \\ 
	p & = p_{0}(z) + p'(x,y,z,t)	\\
	 \PD{p}{z} & = - \rho g \label{eq:hydrostatic}
}
where $p_{0}$ is defined such that $\PD{p_{0}}{z} = -\rho_{0}g$ and with boundary condition $p_{0}(0) = 0$. Then $p_{0}(z) = -g\rho_{0}z$. From this we may deduce (taking atmospheric pressure to be zero) that:
\eqnn{
	p'(x,y,z,t) = g\rho_{0}H(x,y) + g\int_{z}^{H(x,y)}\rho'(z')dz' \approx   g\rho_{0}H(x,y) + g\int_{z}^{0}\rho'(z')dz'
}
Furthermore, \cref{eq:hydrostatic} can be integrated to show that:
\eqnn{
	 \lrb{\int_{A_{+}} - \int_{A_{-}}} p n dS = g\int_{A}\lrb{(-\nabla z_{+}, 1)\int_{z_{+}(x,y)}^{H(x,y)} \rho(z')dz' - (-\nabla z_{-}, 1)\int_{z_{-}(x,y)}^{H(x,y)} \rho(z')dz' }dS
}
where $H(x,y)$ is the height of the upper surface of the fluid, and we take $p(x,y,H) = 0$. The z-component of the above equation is equal to $-g\int_{A} h\ol{\rho} dS$.\\
 Then the horizontal component of our equation of motion becomes:
	\algnn{
	&\D{}{t} \int_{A} h \ol{ \B{u}} dS + \oint_{\pr A} h \ol{ \B{u}  \B{u}}\cdot \B{n} dl + \int_{A} \lrsq{ (\B{u} \tl{w})_{+}\sqrt{1 + \abs{\nabla z_{+}}^{2}} - (\B{u} \tl{w})_{-}\sqrt{1 + \abs{\nabla z_{-}}^{2}} } dS  \\
	& \qquad  = -\int_{A}h f e_{3}\times \ol{\B{u}} dS - \frac{1}{\rho_{0}}\oint_{\pr A} h\ol{p} \B{n} dl + \frac{1}{\rho_{0}}\int_{A} \lrsq{p_{+}\nabla z_{+} - p_{-}\nabla z_{-}}  dS
}
where $p_{\pm} : = p(x,y,z_{\pm},t)$. It may be shown that if we expand $p = p_{0} + p'$ then the terms with $p_{0}$ cancel out and so we can replace $p$ in the above with $p'$. We can also write, as an alternative:
\eqnn{
	(\B{u} \tl{w})_{+}\sqrt{1 + \abs{\nabla z_{+}}^{2}} =  \B{u}_{+}\lrb{-\B{u}_{+}\cdot \nabla z_{+} + (w_{+}-\hat{w}_{+})} =: \B{u}_{+} \phi_{+}
	}
	and similarly with $\B{u}_{-}$. Moreover, we observe that:
\eqnn{
	\oint_{\pr A} h\ol{p'}\B{n} dl = \int_{A} \nabla \lrb{\int_{z_{-}}^{z_{+}} p' dz} dS = \int_{A} \lrb{p'_{+}\nabla z_{+} - p'_{-}\nabla z_{-} + \int_{z_{-}}^{z_{+}}\nabla p dz}
}
	Then the overall equation becomes:
	\algnn{
	&\D{}{t} \int_{A} h \ol{ \B{u}} dS + \oint_{\pr A} h \ol{ \B{u}  \B{u}}\cdot \B{n} dl + \int_{A} \lrsq{ \B{u}_{+} \phi_{+} - \B{u}_{-}\phi_{-} } dS  \\
	& \qquad  = -\int_{A}h f e_{3}\times \ol{\B{u}} dS -  \frac{1}{\rho_{0}}\int_{A} h \ol{\nabla p'} dS
}
Now, to obtain equation (6) from Danilov et al (2016) we follow the procedure of Ringler et al (2013), dividing by the area $|A|$ and taking the small-$A$ limit, which is:
\algn{
	&\PD{}{t}  \lrb{h \ol{ \B{u}}} + \nabla\cdot\lrb{ h \ol{ \B{u}  \B{u}}} +  \B{u}_{+}\phi_{+} - \B{u}_{-}\phi_{-}   = - h f e_{3}\times \ol{\B{u}}  - \frac{1}{\rho_{0}}h\ol{\nabla p'} \label{eq:infinitessimalmomeq}
}

To obtain the alternative form of the momentum equation, we consider the continuity equation. Conservation of mass gives:
\eqnn{
	\D{}{t} \int_{V_{k}(t)} \rho d^{3}x + \int_{\pr V_{k}(t)} \rho u\cdot n dS = 0
}
As before we split $V_{k}(t)$ into its vertical and horizontal components:
\eqnn{
	\D{}{t} \int_{A} h\ol{\rho} dS + \int_{\pr A} h \ol{ \rho \B{u}}\cdot n dS + \int_{A}\lrsq{\rho_{+}\phi_{+} - \rho_{-}\phi_{-}}dS = 0
}
We apply the Boussinesq approximation:
\eqnn{
	\D{}{t} \int_{A} h dS + \int_{\pr A} h \ol{ \B{u}}\cdot n dS + \int_{A}\lrsq{\phi_{+} - \phi_{-}}dS = 0
}
Then, taking the small-$A$ limit:
\eqnn{
	\PD{h}{t} + \nabla\cdot\lrb{ h \ol{ \B{u}}} + \lrsq{\phi_{+} - \phi_{-}} = 0
}
Now if we take \cref{eq:infinitessimalmomeq} and ignore the averaging overbars, then we can combine it with the continuity equation to obtain:
\algnn{
	&\PD{\B{u}}{t}  + \B{u}\cdot\nabla\B{u} +  \Half\lrsq{(\phi \pr_{z}\B{u})_{+} + (\phi\pr_{z}\b{u})_{-}}   \nonumber\\
	& \qquad \qquad  = - f e_{3}\times \B{u}  - \frac{1}{h\rho_{0}}\nabla \lrb{ h p}+ \frac{1}{h\rho_{0}} \lrsq{p_{+}\nabla z_{+} - p_{-}\nabla z_{-}}  
}
Where we have  defined $(\pr_{z}\B{u})_{+} : = \frac{2}{h}\lrsq{\B{u}_{+} - \B{u}}$ and $(\pr_{z}\B{u})_{-} : = \frac{2}{h}\lrsq{\B{u} - \B{u}_{-}}$.

\section{Discretisation}
Here we attempt to discretise \cref{eq:integralmomeq}. First define:
\eqnn{
	\B{U}_{k}(t) : = \frac{1}{\abs{A}}\int_{A} h\ol{\B{u}} dS
}
We think of $\B{U}_{k}(t)$ as the momentum at the centre of the cell $V_{k}(t)$. Further, define $\B{u}^{k}: = \B{U}_{k}/h_{k}$, where $h_{k}$ will be defined later. Then set:
\algnn{
	 \frac{1}{\abs{A}}\int_{\pr A} h \ol{ \B{u}  \B{u}}\cdot \B{n} dS  & \approx \frac{1}{\abs{A}}\sum_{e(k)} l_{e} \B{u}_{e}\B{U}_{e}\cdot \b{n}_{e} \\
	 																							& = \frac{1}{\abs{ A}}\sum_{e(k)} l_{e} \lrb{\Half\sum_{v(e)} \B{u}_{v}}\lrb{\Half\sum_{v(e)}\B{U}_{v}} \cdot \B{n}_{e} \\
	 																							& = \frac{1}{\abs{A}}\sum_{e(k)} l_{e} \lrb{\Half\sum_{v(e)}\frac{1}{3A_{v}} \lrb{\sum_{c(v)} \B{u}_{c}A_{c}}}\lrb{\Half\sum_{v(e)}\frac{1}{3 A_{v}}\lrb{\sum_{c(v)}\B{U}_{c}A_{c}}}  \cdot \B{n}_{e} \\
	 																							& =: \lrsq{\nabla\cdot (\B{U}\B{u})}_{k}(t)
}
Additionally, we approximate:
 \eqnn{
 	\frac{1}{\abs{A}}\int_{A} \lrb{\B{u}_{+} \phi_{+} - \B{u}_{-} \phi_{-} }dS  \approx \lrb{\B{u}^{k}_{+} \phi_{+}^{k} - \B{u}^{k}_{-} \phi_{-}^{k}}
 	}
 	And:
 	\eqnn{
 		\frac{1}{\abs{A}}\int_{A}h f e_{3}\times \ol{\B{u}} dS  \approx f e_{3}\times \B{U}_{k}
 		}
 		Moreover:
 		\algnn{
 			\frac{1}{\abs{A}} \int_{\pr A} h \ol{p}\B{n} dS  & \approx \frac{1}{\abs{A}} \sum_{e(k)}l_{e} (hp)_{e}\B{n}_{e}  \\
 																										 &  =  \frac{1}{\abs{A}} \sum_{e(k)}l_{e} \lrb{\Half\sum_{v(e)}p_{v}h_{v}}\B{n}_{e}  \\
 																										 &  =: \lrsq{\nabla(hp)}_{k}
 		}
 		Finally, setting:
 		\algnn{
 			\frac{1}{\abs{A}}\lrsq{p_{+}\nabla z_{+} - p_{-}\nabla z_{-}} & \approx p_{+}\nabla z_{+} - p_{-}\nabla z_{-} 
 				}
 			Overall, we have:
 			\algn{
 				& \D{\B{U}_{k}}{t} + \lrsq{\nabla\cdot(\B{U}\B{u})}_{k} + \B{u}^{k}_{+} \phi_{+}^{k} - \B{u}^{k}_{-} \phi_{-}^{k} + f e_{3}\times \B{U}_{k} = -\frac{1}{\rho_{0}}\lrsq{\nabla(hp)}_{k} + \frac{1}{\rho_{0}}\lrsq{p_{+}^{k}\nabla z_{+}^{k} - p_{-}^{k}\nabla z_{-}^{k}} \label{eq:discretemomeq}
 			}
 			Now, if we formally expand the right-hand side we obtain:
 			\algnn{
 				-(\nabla h)p - h\nabla p + p_{+}\nabla z_{+} - p_{-}\nabla z_{-} & = 
 			}
 			And the continuity equation is obtained by the following designations:
 			\algnn{
 					\frac{1}{\abs{A}}\int_{A} h dS  & =: h_{k}  \\
 					\frac{1}{\abs{A}} \int_{\pr A} h\ol{\B{u}}\cdot \B{n} dS  & \approx \sum_{e(k)} l_{e} \B{U}_{e}\cdot \B{n}_{e} \\ 
 																								& = \sum_{e(k)} l_{e} \lrb{\Half\sum_{v(e)}\frac{1}{3 A_{v}}\lrb{\sum_{c(v)}\B{U}_{c}A_{c}}}  \cdot \B{n}_{e} \\
 																								& =: \lrsq{\nabla \cdot\lrb{h\B{u}}}_{k} \\
 					\frac{1}{A} \int_{A}\lrsq{\phi_{+} - \phi_{-}}dS & \approx \phi_{+}^{k} - \phi_{-}^{k}
 			}
 			Combining these gives:
 			\eqlab{
 				\D{h_{k}}{t} + \lrsq{\nabla \cdot\lrb{h\B{u}}}_{k} + \phi_{+}^{k} - \phi_{-}^{k} = 0
 			}{discreteconteq}
 			Summing \cref{eq:discreteconteq} over vertical layers gives:
 			\eqlab{
 				\D{\eta}{t} + \sum_{k}\lrsq{\nabla \cdot\lrb{h\B{u}}}_{k} +\phi_{\text{top}} = 0
 			}{totalconteq}
 			where $\eta$ is the total height of the fluid layer and $\phi_{top}$ the flux through the fluid surface.
 			
 			\section{Time Stepping}
 			In Danilov et al (2016) they use the following evolution equation for $\B{u}$ (need more detail on how to obtain this):
 			\algnn{
 				& \pr_{t}\B{u} + \frac{\omega + f}{h} e_{3}\times \B{u}h + \Half\lrsq{\lrb{w\pr_{z}\B{u}}_{+} - \lrb{w\pr_{z}\B{u}}_{-}} + \nabla\lrb{p/\rho_{0} + \Half\abs{\B{u}}^{2}} + g\rho\nabla Z/\rho_{0}  = D_{u}(\B{u}) + \pr_{z}\lrb{\nu_{\text{v}}\pr_{z}\B{u}}
 			}
 			with
 			\eqnn{
 				p = g \rho_{0} \eta + g\int_{z}^{0} \rho(z') dz'
 			}
 			Thus we may write the evolution equation as:
 			\eqnn{
 				\pr_{t}\B{u} = \B{R}_{u} + \pr_{z}\lrb{\nu_{\text{v}}\B{u}} - g\nabla\eta
 			}
 			where
 			\eqnn{
 				\B{R}_{u} : = D_{u}(\B{u})  - \lrb{ \frac{\omega + f}{h} e_{3}\times \B{u}h + \Half\lrsq{\lrb{w\pr_{z}\B{u}}_{+} - \lrb{w\pr_{z}\B{u}}_{-}} + \nabla\lrb{p_{h}/\rho_{0} + \Half\abs{\B{u}}^{2}} + g\rho\nabla Z/\rho_{0}}
 			}
 			and $p_{h} = g\int_{z}^{0} \rho(z') dz'$.\\
 			\linebreak
 			In summary we need to solve:
 			\algnn{
 				&\pr_{t}\B{u}_{k}  = \B{R}_{u,k} + \pr_{z}\lrb{\nu_{\text{v}}\B{u}_{k}} - g\nabla\eta \\
 				&\D{h_{k}}{t} + \lrsq{\nabla \cdot\lrb{h\B{u}}}_{k} + \phi_{+}^{k} - \phi_{-}^{k} = 0 \\
 				&\D{\eta}{t} + \sum_{k}\lrsq{\nabla \cdot\lrb{h\B{u}}}_{k} +\phi_{\text{top}} = 0
 			}
 			To do this we discretise using asynchronous time-stepping:
 			\eq{
 				h^{n+1/2}_{k} - h^{n-1/2}_{k} = - \Delta t\lrsq{\nabla\cdot\lrb{\B{u}^{n}_{k}h^{n-1/2}_{k}} + \phi_{+}^{k} - \phi_{-}^{k}}
 			}
 			then with $h$ defined in this way, let:
 			\eqnn{
 				\bar{h} : = \sum_{k}h_{k} - H
 			}
 			$\eta$ and $\bar{h}$ should be the same, but they differ because of the temporal discretisation used. Therefore we need a scheme in which they do not diverge too much. \\
 			\linebreak
 			$\eta$ evolves according to:
 			\eqnn{
 				\eta^{n+1} - \eta^{n} = -\Delta t\lrsq{ \al\lrb{\nabla\cdot \int^{\bar{h}^{n+1/2}}\B{u}^{n+1}dz + \phi^{n+1/2}_{\text{top}}} + (1-\al)\lrb{\nabla\cdot \int^{\bar{h}^{n - 1/2}}\B{u}^{n}dz + \phi_{\text{top}}^{n-1/2}} } 
 			}
 			where $\al$ is an implicitness parameter and the integral is really a sum vertically over layers.\\
 			\linebreak
 			On the other hand, $\bar{h}$ evolves according to:
 			\eqnn{
 				\bar{h}^{n+1/2} - \bar{h}^{n-1/2} = -\Delta t \lrsq{\nabla\cdot \int^{\bar{h}^{n-1/2}}\B{u}^{n}dz - \phi_{\text{top}}^{n-1/2}}	
 			}
 			The evolution equations for $\bar{h}$ and $\eta$ are therefore consistent if we require:
 			\eqnn{
 				\eta^{n} = \al \bar{h}^{n+1/2} + (1-\al)\bar{h}^{n-1/2}
 			}
 			To discretise the momentum equation we write:
 			\eqnn{
 				\B{u}^{n+1}_{k} - \B{u}^{n+1}_{k} = \Delta t \lrsq{\B{R}_{u,k}^{n+1/2} + \pr_{z}\nu_{v}\pr_{z}\B{u}^{n+1}_{k} - g\nabla\lrb{\theta \eta^{n+1} + (1-\theta)\eta^{n}}}
 			}
 			and we use the Adams-Bashforth method to calculate $\B{R}_{u,k}^{n+1/2}$. However, we split this into two steps using a predictor-corrector method:
 			\algnn{
 				\B{u}^{*}_{k} - \B{u}^{n}_{k} & = \Delta t\lrsq{\pr_{z}\nu_{v}\pr_{z}\lrb{\B{u}^{*}_{k} - \B{u}^{*}_{k}} + \B{R}_{u,k}^{n+1/2} + \pr_{z}\nu_{v}\pr_{z}\B{u}^{n}_{k} - g\nabla\eta^{n}} \\
 				\B{u}^{n+1}_{k} - \B{u}^{*}_{k} & = -g\Delta t \theta \nabla\lrb{\eta^{n+1} - \eta^{n}}
 			}
 			So if, after the first step we define $\Delta \B{u} : = \b{u}^{*} - \B{u}^{n}$, then the $\eta$ equation becomes:
 			\algnn{
 				\eta^{n+1} - \eta^{n} & - \al\theta g \Delta t^{2} \nabla\cdot \int^{\bar{h}^{n+1/2}}\nabla\lrb{\eta^{n+1} - \eta^{n}}dz \\ 
 					& = -\Delta t \al\lrb{\nabla\cdot\int^{\bar{h}^{n+1/2}}(\B{u}^{n} + \Delta \B{u})dz + \phi^{n+1/2}_{\text{top}}} - \Delta t(1-\al)\lrb{\nabla\cdot \int^{\bar{h}^{n-1/2}}\B{u}^{n}dz + \phi^{n-1/2}_{\text{top}}}
 			}
 			Overall, then, the strategy is as follows:
 			\begin{itemize}
 				\item $\eta^{n} = \al \bar{h}^{n+1/2} + (1-\al)\bar{h}^{n-1/2}$.
 				\item Calculate $\Delta \B{u}$ from $\B{u}^{*}_{k} - \B{u}^{n}_{k}  = \Delta t\lrsq{\pr_{z}\lrb{\nu_{v}\pr_{z}\lrb{\B{u}^{*}_{k} - \B{u}^{n}_{k}}} + \B{R}_{u,k}^{n+1/2} + \pr_{z}\nu_{v}\pr_{z}\B{u}^{n}_{k} - g\nabla\eta^{n}}$.
 				\item Evolve $\eta$ by:
 					\algnn{
 				\eta^{n+1} - \eta^{n} & - \al\theta g \Delta t^{2} \nabla\cdot \int^{\bar{h}^{n+1/2}}\nabla\lrb{\eta^{n+1} - \eta^{n}}dz \\ 
 					& = -\Delta t \al\lrb{\nabla\cdot\int^{\bar{h}^{n+1/2}}(\B{u}^{n} + \Delta \B{u})dz + \phi^{n+1/2}_{\text{top}}} - \Delta t(1-\al)\lrb{\nabla\cdot \int^{\bar{h}^{n-1/2}}\B{u}^{n}dz + \phi^{n-1/2}_{\text{top}}}
 			}
 			\item Get the new horizontal velocity from:
 				\eqnn{
 					\B{u}^{n+1}_{k} - \B{u}^{*}_{k} = -g\Delta t \theta \nabla\lrb{\eta^{n+1} - \eta^{n}}
 				}
 				\item Evolve $\bar{h}$ according to:
 						\eqnn{
 				\bar{h}^{n+3/2} - \bar{h}^{n+1/2} = -\Delta t \lrsq{\nabla\cdot \int^{\bar{h}^{n+1/2}}\B{u}^{n+1}dz - \phi_{\text{top}}^{n+1/2}}	
 			}
 			\item Determine new layer thicknesses. We do this by distributing the total change in height over all eligible layers:
 						\eqnn{
 							\pr_{t} h_{k} = \lrb{h^{0}_{k}/\tl{H}} \pr_{t}\bar{h}
 						} 
 						where $\tl{H}$ is the unperturbed height of the eligible layers and $h^{0}_{k}$ are arbitrary heights which sum to $\tl{H}$.
 			\item We may then calculate:
 				\eqnn{
 					(\phi^{k}_{+})^{n+3/2} -( \phi^{k}_{-} )^{n+3/2}= - \frac{h_{k}^{n+3/2} - h_{k}^{n+1/2}}{\Delta t} - \nabla\cdot\lrb{\B{u}^{n+1}_{k}h^{n+1/2}_{k}} 
 				}
 				to get the fluxes $\phi_{\pm}$, using the fact that at the bottom layer the flux $\phi_{\text{bottom}} = 0$.
 				\item Finally, we advance the tracers. The tracer equation is written as:
 					\eqnn{
 						\pr_{t}(hT)_{k} + \nabla\cdot(\B{u}hT)_{k} + \lrb{\phi_{+}T_{+} - \phi_{+}T_{+}}_{k}  = \nabla_{3}\cdot\lrb{K\nabla_{3}T_{k}}
 					}
 					where $K$ is the diffusion tensor, a $3\times 3$ matrix. We can also write the above equation as:
 					\eqnn{
 						\pr_{t}(hT)_{k}  =R_{T} +  \pr_{z}\lrb{h K_{33}\pr_{z}T_{k}}
 					}		
 					which we discretise as:
 					\eqnn{
 						h^{n+1/2}_{k}T^{n+1/2}_{k} - h^{n-1/2}_{k}T^{n-1/2}_{k}  = \Delta t\lrsq{R_{T}^{n} + \lrb{K_{33}\pr_{z}T_{k}}^{+} - \lrb{K_{33}\pr_{z}T_{k}}^{-}}
 					}
 					and split into two steps:
 					\algnn{
 						h^{n+1/2}_{k}T^{*}_{k} - h^{n-1/2}_{k}T^{n-1/2}_{k}  &= \Delta t R_{T}^{n} \\
 						h^{n+1/2}_{k}\lrb{T^{n+1/2}_{k} - T^{*}_{k}}  &= \Delta t\lrsq{ K_{33}\pr_{z}(T_{k}^{n+1/2}-T^{*}_{k}) + K_{33}\pr_{z}T^{*}}_{-}^{+}
 					}
 			\end{itemize}
 			
\end{document}