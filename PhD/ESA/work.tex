\begin{itemize}
	\item Derivation of equations of motion.
	\item Description of solving scheme.
	\item Model configuration.
	\item Results of various runs and analysis. 
\end{itemize}

\begin{itemize}
	\item Ocean plays important role in climate: gulf stream determines climate of Europe.
	\item However, mechanism for maintaining of Gulf Stream is poorly understood--large number of degrees of freedom.
	\item Small-scale eddy effect thought to be responsible for the Gulf Stream.
	\item In order to correctly represent Gulf Stream in numerical models it is necessary to include these small scale effects.
	\item However, simulations on this scale are impracticable for climate studies. 
	\item Therefore it is necessary to develop parameterisations which account for the eddy effects in some sense without resolving them explicitly.
	\item Previous attempts include Leith, Smagorinsky, Gent-McWilliams. These are usually based on heuristic arguments involving energy. 
	\item Other methods include parameterisation by dynamic forcing (Berloff), stochastic transport (Holm et al) and Data-Driven methods. These have been shown to be effective in the QG case but it remains to demonstrate their applicability to Primitive Equation Models. 
\end{itemize}

\begin{itemize}
	\item Derivation of primitive equations.	
	\item Description of differences from exact equations and QG, RSW.
\end{itemize}

\iffalse
In a general setting, we consider a fluid of three-dimensional velocity field $u$ and density field $\rho$ contained in an arbitrary volume $V$ and acted upon by some external forces per unit volume $F$ and surface forces per unit area $\tau n$.  Then by Newton's Second Law we have the change in total momentum given by:
\eqnn{
	\D{}{t} \int_{V} \rho u d^{3}x = \int_{V} F d^{3}x + \int_{\pr V} \tau n dS - \int_{\pr V} \rho u u\cdot n dS
}
The final term is the momentum flux through the boundary of $V$. For a rotating fluid in a gravitational field we take $F$ to include the Coriolis force and gravitational force, while $\tau$ accounts for the pressure and viscosity terms.  Explicitly we take $F = -\rho g \hat{k} -\rho f \hat{k}\times u$, where $f = 2\Omega\cos(y)$ is the Coriolis parameter and $\tau_{ij} = -p(x,t)\delta_{ij} + \sigma_{ij}(u)$, where $p$ is pressure and $\sigma_{ij}(u)$ is some tensor accounting for viscosity, to be specified later. Using the divergence theorem to get all terms under a volume integral, and using the fact that the volume of integration is arbitrary, we arrive at the following form of the momentum equation:
\eqlab{
	\pr_{t}(\rho u) + \nabla\cdot(\rho u\otimes u) +  \rho f\hat{k}\times u + \rho g \hat{k} = -\nabla p + \nabla\cdot \sigma
}{momeq}
Following a similar procedure we may derive the equation of conservation of mass:
\eqlab{
	\D{}{t}\int_{V}\rho d^{3}x = - \int_{\pr V} \rho u\cdot n dS\quad  \implies \quad \pr_{t}\rho + \nabla\cdot(\rho u) = 0 
}{conteq}
We may then use \cref{eq:conteq} to simplify \cref{eq:momeq} and get:
\eqlab{
	\pr_{t} u + u\cdot\nabla u + f\hat{k}\times u + g\hat{k} = -\frac{1}{\rho}\nabla p + \frac{1}{\rho}\nabla\cdot\sigma
}{veleq}
We also need a further equation to tell us how to calculate $p$. We can do this via an equation of state $\rho = \rho (p, T, S)$, where $T$ is temperature and S represents other tracers such as salinity or entropy; however, we only consider temperature here so that $\rho = \rho(p,T)$. Temperature evolves according to the tracer advection equation (see GFD lecture notes):
\eqlab{
	\pr_{t} (\rho T) + \nabla\cdot(\rho u T) = \nabla\cdot(\kappa\nabla T)
}{traceq}
The set of equations we have to solve is then \cref{eq:conteq,eq:veleq,eq:traceq} together with $\rho = \rho(p,T)$. These are the complete fluid equations. \\
\linebreak
However, to arrive at the primitive equation approximation, let us non-dimensionalize \cref{eq:veleq,eq:conteq}:
	\algnn{
		\eps\lrb{\pr_{t} \B{u} + \B{u}\cdot\nabla \B{u} + w\PD{w}{z}} + f\hat{k}\times \B{u}  & = -\frac{1}{(1+b)}\nabla p + \frac{1}{1+b}\nabla\cdot\sigma \\
		\eps\al^{2}\lrb{\pr_{t}w + \B{u}\cdot\nabla w + w\PD{w}{z}} &= -\frac{1}{1+b}\PD{p}{z} - \frac{1}{\Fc\eps} \\
		\pr_{t}b + (1+b)\lrb{\nabla\cdot \B{u} + \PD{w}{z}} + \B{u}\cdot\nabla b + w\PD{b}{z} &= 0 
	}
	where $\al = H/L$, $\rho = \rho_{0}(1+b)$.
We then make the following assumptions:
\begin{itemize}
	\item $\al = H/L \ll 1$. The consequence of this is that the vertical velocity equation satisfies, to leading order, the hydrostatic balance equation $\PD{p}{z} = -\rho g$. 
	\item The variations in density are small, $\rho(x,t) = \rho_{0}\lrb{1 + b(x,t)}$ with $b\ll 1$. 
	\item The flow is divergence-free $\nabla\cdot u = 0$. \footnote{\url{https://ocw.mit.edu/courses/earth-atmospheric-and-planetary-sciences/12-950-atmospheric-and-oceanic-modeling-spring-2004/lecture-notes/lec20.pdf}}. That is, we ignore the terms involving $b$ in the continuity equation. 
\end{itemize}
With these assumptions \cref{eq:veleq} becomes:
\eqlab{
	\pr_{t}\B{u} + \B{u}\cdot\nabla\B{u} + w\PD{\B{u}}{z} + f\hat{k}\times\B{u} = - \frac{1}{\rho_{0}} \nabla p' + \frac{1}{\rho_{0}}\nabla\cdot \sigma
}{horizveleq}
\eqlab{
	p'(\B{x},z,t) = \rho_{0}g\eta + g\int_{z}^{\eta}\rho'(\B{x},z',t)dz'
}{pressureq}
where $p = p_{0} + p'$ and $p_{0}$ is defined by $\PD{p_{0}}{z} = -\rho_{0}g$ and $\left. p_{0}\right|_{z=0} = 0$. $\eta$ is the surface height of the fluid\\
\linebreak
Moreover, the divergence-free assumption means that the continuity equation is replaced with:
\eqlab{
	\nabla\cdot\B{u} + \PD{w}{z} = 0
}{diveq}
Furthermore, if we integrate the above equation in the vertical direction between the bottom topography $-H$ and the sea-surface height $\eta$ then we find:
\eqlab{
	\pr_{t}\eta + \nabla\cdot \int_{-H}^{\eta}\B{u}dz = 0
}{ssheq}
Finally, the tracer equation becomes:
\eqlab{
	\pr_{t}T + \nabla\cdot(\B{u}T) + \PD{(wT)}{z} = \frac{1}{\rho_{0}}\nabla\cdot(\kappa\nabla T)
}{traceqnew}
The primitive equations are \cref{eq:horizveleq,eq:pressureq,eq:diveq,eq:ssheq,eq:traceqnew} along with the equation of state $\rho' = \rho'(p',T)$. 

\subsection{Derivation from a variational principle}

We may also derive the primitive equations from a variational principle by considering the following action (reference Holm):
\eqnn{
	S = \int_{0}^{T}l(\B{u},w,D,b)dt = \int \lrc{\frac{\eps}{2}D u\cdot\Lambda u + D u \cdot R - Dbz - p(D-1)}d^{2}x dz dt
}
where $\Lambda = \diag(1,1,\eps^{2}\al^{2})$; $b = g\rho/\rho_{0}$. $u$ is the three-dimensional velocity and $D,b$ are advected  which we assume to satisfy the advection equation:
\eqnn{
	\pr_{t}D + \L_{u}D = 0 \qquad \pr_{t}b + \L_{u}b = 0  
}
The lagrangian $l$ satisfies the Euler-Poincar\'{e} equation:
\eqnn{
	\pr_{t}\lrb{\dD{l}{u}} + \L_{u}\lrb{\dD{l}{u}} = \dD{l}{D}\diamond D + \dD{l}{b}\diamond b
}
where the diamond is defined by $\EXP{p\diamond q, v} = -\EXP{p,\L_{v}q}$. Using the advection equations we can re-write the above as:
\eqnn{
	\pr_{t}\lrb{\frac{1}{D}\dD{l}{u}} + \L_{u}\lrb{\frac{1}{D}\dD{l}{u}} = \frac{1}{D}\lrb{\dD{l}{D}\diamond D +\dD{l}{b}\diamond b}
}
Substituting in the appropriate expressions then gives:
\eqnn{
	\pr_{t}\lrb{\frac{1}{D}\dD{l}{u}} + u\cdot\nabla\lrb{\frac{1}{D}\dD{l}{u}} + \nabla u \cdot \lrb{\frac{1}{D}\dD{l}{u}} = \nabla\lrb{\dD{l}{D}} - \frac{1}{D}\dD{l}{b}\nabla b
}
we may then use:
\algnn{
	\frac{1}{D}\dD{l}{u} &= \eps \Lambda u + R \\
	 \dD{l}{D} &= \frac{\eps}{2} u\cdot\Lambda u + u \cdot R - bz - p \\
	 \frac{1}{D}\dD{l}{b} & = -z 
}
Inserting these values, we find:
\algnn{
	\eps\lrb{\pr_{t}\B{u} + \B{u}\cdot\nabla\B{u} + w\PD{\B{u}}{z}} + fe_{3}\times\B{u} &= -\nabla p \\
	\eps^{3}\al^{2}\lrb{\pr_{t}w + \B{u}\cdot\nabla w + w\PD{w}{z}} = -\PD{p}{z} - b
}
We also have the auxiliary equations given by advection. These, combined with the Lagrange multiplier equation $D=1$ give:
\algnn{
	\nabla\cdot\B{u} + \PD{w}{z} &= 0 \\
	\pr_{t}b + \B{u}\cdot\nabla b + w\PD{b}{z} &= 0
}
Now, to derive the primitive equations we take the limit $\al\arr 0$. 
\fi

