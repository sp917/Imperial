Large scale western-boundary currents in the ocean play a significant role in the Earth's climate. The Gulf Stream, for example, transports warm water across the Atlantic and thus is largely responsible for the relatively mild climate in Europe \cite{rossby_1996}. However, the exact mechanism by which such boundary currents are formed and maintained is poorly understood, though it is thought that the small-scale features of the ocean, referred to as eddies, have an important effect on such large-scale features like the Gulf Stream \cite{berloff_2015}. Therefore, in order to accurately model a large-scale western-boundary current in a computer simulation, it is necessary to use a grid resolution sufficiently small as to be able to resolve the eddy effects responsible for maintaining the current. However, for simulations on time-scales relevant to climate studies, running a model with such high resolution is computationally infeasible. \\
\indnt In practice then, we must run global ocean models at a relatively low resolution; not only does this cause the eddy effect to go unresolved or only partially resolved, but additionally the viscosity will dissipate energy at a much greater rate than is physically realistic \cite{juricke_danilov_kutsenko_oliver_2019}. It is therefore common practice to introduce a so-called `eddy parameterisation'. This is commonly done by means of a Gent-McWilliams parameterisation \cite{gent_mcwilliams_1990}, and momentum closures such as those proposed by Leith, Smagorinski \cite{fox-kemper_menemenlis_2008} or Jansen-Held \cite{jansen_held_2014}. However, many of these methods are derived heuristically based on energetic considerations and do not have a solid theoretical justification beyond quasigeostrophic theory \cite{danilov_juricke_kutsenko_oliver_2019}. \\
\indnt  Therefore it is an important active area of research to try to implement new parameterisation schemes that should allow for a more realistic representation of small scale eddy effects in ocean circulation models. There have been several proposed possibilities for this: Berloff \cite{berloff_2015,berloff_2016,berloff_2018} proposed using the statistics of a high-resolution simulation to construct forcings which can be used to represent eddy effects in low-resolution simulations. Cotter et al \cite{cotter_crisan_holm_pan_shevchenko_2018} suggest the introduction of a stochastic transport term to the equations of motion in such a way that the small-scale effects are represented by stochastic noise, while still preserving the fundamental properties of the equations such as energy conservation and the Kelvin circulation property.  There are also data-driven approaches such as that proposed in \cite{kondrashov_chekroun_berloff_2018}. \\
\indnt The above methods have been tested on simple quasi-geostrophic models; however, realistic global ocean models used for climate studies use the Primitive Equations or Boussinesq Equations. Therefore, in order to evaluate the effectiveness of the newly proposed parameterisations we plan to adapt and implement them into Primitive Equation ocean models. The models we choose are FESOM2.0, a finite-volume model with a discretisation of the spatial domain by triangular prisms, and MITgcm, which also uses a finite volume method but with a square discretisation. \\
\indnt In order to focus on the properties of the parameterisation schemes, and to reduce the computational complexity, we shall not initially employ the full global-ocean model capabilities of these models, but rather focus on a simplified domain consisting of a rectangular box between latitudes $30^{\circ}$N and $60^{\circ}$N, representing the North Atlantic; a wind-forcing is imposed to give rise to a double gyre. The initial aim is to configure these models so that they produce a western-boundary current when run with high-resolution. We shall then be able to implement the parameterisations in low-resolution runs and observe their effects. 