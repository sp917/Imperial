 \documentclass[10pt]{article}

\usepackage{amsfonts}
\usepackage{amsthm}
\usepackage[mathscr]{euscript}
\usepackage{amsmath}
\usepackage{amssymb}
\usepackage{tensor}
\usepackage{fullpage}
\usepackage{enumerate}
\usepackage{graphicx,caption}
\usepackage{float}
\usepackage{url}
\usepackage[hidelinks]{hyperref}
\usepackage{cleveref}
\usepackage[font=footnotesize,labelfont=footnotesize]{caption}
\usepackage[ansinew]{inputenc}
\usepackage{pbox}
\usepackage{cancel}
\usepackage{accents}
\usepackage{subcaption}
\usepackage{color}


\setlength\parindent{0pt}
\linespread{1.2}

\newcommand{\D}[2]{\frac{d #1}{d #2}}
\newcommand{\PD}[2]{\frac{\partial #1}{\partial #2}}
\newcommand{\eq}[1]{\begin{equation} #1 \end{equation}}
\newcommand{\eqlab}[2]{\begin{equation} #1 \label{eq:#2} \end{equation}}
\newcommand{\eqnn}[1]{\eq{#1 \nonumber}}
\newcommand{\algn}[1]{\begin{align} #1 \end{align}}
\newcommand{\algnn}[1]{\begin{align*} #1 \end{align*}}
\newcommand{\bra}[1]{\left\langle #1 \right|}
\newcommand{\ket}[1]{\left| #1 \right\rangle}
\newcommand{\braket}[2]{\langle #1 | #2 \rangle}
\newcommand{\Braket}[3]{\langle #1 | #2 | #3 \rangle}
\newcommand{\B}[1]{\mathbf{#1}}
\newcommand{\tvec}[2]{\left(\begin{array}{c} #1 \\ #2 \end{array}\right)}
\newcommand{\tvecs}[2]{\left(\begin{smallmatrix} #1\\#2\end{smallmatrix}\right)}
\newcommand{\twobytwomatrix}[4]{\left(\begin{array}{cc} #1 & #2 \\ #3 & #4\end{array}\right)}
\newcommand{\threevec}[3]{\left(\begin{array}{c} #1 \\ #2 \\ #3 \end{array}\right)}
\newcommand{\ph}{\varphi}
\newcommand{\TL}[1]{\tensor{\Lambda}{#1}}
\newcommand{\TG}[1]{\tensor{\Gamma}{#1}}
\newcommand{\TT}[1]{\tensor{T}{#1}}
\newcommand{\TA}[1]{\tensor{A}{#1}}
\newcommand{\Td}[1]{\tensor{\delta}{#1}}
\newcommand{\TAinv}[1]{\tensor{{(A^{-1})}}{#1}}
\newcommand{\M}{\mathcal{M}}
\newcommand{\N}{\mathcal{N}}
\newcommand{\Dc}{\mathcal{D}}
\renewcommand{\L}{\mathcal{L}}
\newcommand{\intp}[1]{\int{\frac{d^{3}\B{#1}}{(2\pi)^{3}}\:}}
\newcommand{\intE}[1]{\int{\frac{d^{3}\B{#1}}{(2\pi)^{3}}\frac{1}{2E_{#1}}\:}}
\newcommand{\intpi}[1]{\int{\frac{d^{3}\B{#1}}{(2\pi)^{3}}\left(-\frac{i}{2}\right)\:}}
\newcommand{\intpq}{\int{\frac{d^{3}\B{p}d^{3}\B{q}}{(2\pi)^{3}(2\pi)^{3}}\:}}
\newcommand{\intEpq}{\int{\frac{d^{3}\B{p}d^{3}\B{q}}{(2\pi)^{3}(2\pi)^{3}}\frac{1}{2E_{p}2E_{q}}\:}}
\newcommand{\inttt}[1]{\int{d^{3}\B{#1}\:}}
\newcommand{\intx}[1]{\int{d#1\:}}
\newcommand{\intiv}[1]{\int{d^{4}#1\:}}
\newcommand{\dotp}[2]{\B{#1}\cdot\B{#2}}
\newcommand{\R}{\mathbb{R}}
\newcommand{\arr}{\rightarrow}
\newcommand{\phipb}{\phi^{*}}
\newcommand{\phipf}{\phi_{*}}
\renewcommand{\b}[1]{\textbf{#1}}
\renewcommand{\i}[1]{\textit{#1}}
\newcommand{\Tc}{\mathcal{T}}
\newcommand{\Ts}[2]{\mathcal{T}_{#1}(\mathcal{#2})}
\newcommand{\Ds}[2]{\mathcal{T}^{*}_{#1}(\mathcal{#2})}
\renewcommand{\O}{\mathcal{O}}
\newcommand{\ol}[1]{\overline{#1}}
\newcommand{\ul}[1]{\underline{#1}}
\newcommand{\Dd}{\mathcal{D}}
\newcommand{\half}{\tfrac{1}{2}}
\newcommand{\bs}[1]{\boldsymbol{#1}}
\newcommand{\tl}[1]{\tilde{#1}}
\newcommand{\lorentz}{\mathbb{L}}
\newcommand{\matr}[2]{\left(\begin{array}{#1} #2\end{array}\right)}
\newcommand{\ee}{{\'e}}
\newcommand{\PP}{\mathbb{P}}
\newcommand{\EXP}[1]{\left\langle #1 \right\rangle}
\newcommand{\Tp}[1]{\tensor{\perp}{#1}}
\newcommand{\TRc}[1]{\tensor{\mathcal{R}}{#1}}
\newcommand{\TR}[1]{\tensor{R}{#1}}
\newcommand{\TK}[1]{\tensor{\kappa}{#1}}
\newcommand{\TN}[2]{\tensor{N}{_{#1}^{#2}}}
\newcommand{\TNs}[2]{\tensor{{N^{*}}}{_{\dot{#1}}^{\dot{#2}}}}
\newcommand{\dal}{\dot{\alpha}}
\newcommand{\dbe}{\dot{\beta}}
\newcommand{\ins}{\int{\frac{d\phi d\psi_{1}d\psi_{2}}{\sqrt{2\pi}}\:}}
\newcommand{\diraceq}{(i\cancel{\partial}-m)\psi}
\newcommand{\conjdirac}{\ol{\psi}(-\overleftarrow{\cancel{\partial}}-m)}
\newcommand{\cp}{\cancel{\partial}}
\newcommand{\gam}[1]{\gamma^{#1}}
\newcommand{\gamu}{\gam{\mu}}
\newcommand{\ganu}{\gam{\nu}}
\newcommand{\gamo}{\matr{cc}{0&I_{2}\\I_{2}&0}}
\newcommand{\gami}{\matr{cc}{0 &\sigma^{i}\\-\sigma{i}&0}}
\newcommand{\can}[1]{\cancel{#1}}
\newcommand{\pr}{\partial}
\newcommand{\PL}{\frac{1}{2}(1-\gam{5})}
\newcommand{\PR}{\frac{1}{2}(1+\gam{5})}
\newcommand{\psib}{\ol{\psi}}
\newcommand{\chib}{\ol{\chi}}
\newcommand{\cD}{\cancel{D}}
\newcommand{\hP}{\hat{P}}
\newcommand{\hT}{\hat{T}}
\newcommand{\poinc}{Poincar{\'e} }
\newcommand{\schro}{Schr{\"o}dinger }
\newcommand{\Pc}{\mathcal{P}}
\newcommand{\itmze}[1]{\begin{itemize} #1 \end{itemize}}
\newcommand{\tM}{\tilde{\M}}
\newcommand{\Lg}{\mathfrak{g}}
\newcommand{\Tf}[1]{\tensor{f}{#1}}
\newcommand{\PB}[2]{\left\{#1,#2\right\}_{PB}}
\newcommand{\nF}{(-)^{F}}
\newcommand{\hC}{\hat{C}}
\newcommand{\schw}{\left(1-\frac{2M}{r}\right)}
\newcommand{\schwarzchild}{ds^{2}=-\schw dt^{2}+\schw^{-1}dr^{2}+r^{2}d\Omega^{2}}
\newcommand{\Dxi}{\D{}{x^{i}}}
\newcommand{\Dxj}{\D{}{x^{j}}}
\newcommand{\dxi}{dx^{i}}
\newcommand{\dxj}{dx^{j}}
\newcommand{\Hc}{\mathcal{H}}
\newcommand{\inttsig}{\int{dt\:}\oint{d\sigma\:}}
\newcommand{\dD}[2]{\frac{\delta #1}{\delta #2}}
\newcommand{\lrarr}{\leftrightarrow}
\newcommand{\dX}{\dot{X}}
\newcommand{\Xd}{{X'}}
\newcommand{\enumi}[1]{\begin{enumerate}[(i).]#1\end{enumerate}}
\newcommand{\Uc}{\mathcal{U}}
\newcommand{\TB}[1]{\tensor{B}{#1}}
\newcommand{\lrb}[1]{\left(#1\right)}
\newcommand{\lrsq}[1]{\left[#1\right]}
\newcommand{\lrc}[1]{\left\{#1\right\}}
\newcommand{\intt}{\int{dt\:}}
\newcommand{\tal}{\tilde{\alpha}}
\newcommand{\al}{\alpha}
\newcommand{\piT}{\pi T}
\newcommand{\sqfr}[2]{\sqrt{\frac{#1}{#2}}}
\newcommand{\frsqi}[2]{\frac{\sqrt{#1}}{#2}}
\newcommand{\frsqii}[2]{\frac{#1}{\sqrt{#2}}}
\newcommand{\tL}{\tilde{L}}
\newcommand{\Half}{\frac{1}{2}}
\newcommand{\Z}{\mathbb{Z}}
\newcommand{\intpisig}{\int_{0}^{\pi}d\sigma\:}
\newcommand{\vac}{\ket{0}}
\newcommand{\phys}{\ket{\text{phys}}}
\newcommand{\hal}{\hat{\alpha}}
\newcommand{\hN}{\hat{N}}
\newcommand{\hA}{\hat{A}}
\newcommand{\TP}[1]{\tensor{P}{#1}}
\newcommand{\TBh}[1]{\tensor{\hat{B}}{#1}}
\newcommand{\hS}{\hat{S}}
\newcommand{\Tab}[1]{\tensor{#1}{^a_b}}
\newcommand{\omh}{\hat{\omega}}
\newcommand{\sigh}{\hat{\sigma}}
\newcommand{\Bh}{\hat{B}}
\newcommand{\hB}{\Bh}
\newcommand{\Braq}[1]{\Braket{q_{f}}{#1}{q_{i}}}
\newcommand{\inttx}{\intx{^3\B{x}}}
\newcommand{\pb}[1]{\{#1\}_{\text{PB}}}
\newcommand{\delal}{\delta_{\alpha}}
\newcommand{\delxi}{\delta_{\xi}}
\newcommand{\deleps}{\delta_{\epsilon}}
\newcommand{\intsig}{\oint d\sigma\:}
\newcommand{\bal}{\bs{\al}}
\newcommand{\tbal}{\tilde{\bal}}
\newcommand{\C}{\mathbb{C}}
\newcommand{\Qc}{\mathcal{Q}}
\newcommand{\TF}[1]{\tensor{F}{#1}}
\newcommand{\scri}{\mathcal{I}}
\newcommand{\scrip}{\scri^{+}}
\newcommand{\scrim}{\scri^{-}}
\newcommand{\Bc}{\mathcal{B}}
\newcommand{\Wc}{\mathcal{W}}
\newcommand{\Ac}{\mathcal{A}}
\newcommand{\Up}{U^{+}}
\newcommand{\Vp}{V^{+}}
\newcommand{\Um}{U^{-}}
\newcommand{\Vm}{V^{-}}
\newcommand{\Mg}{(\M,g)}
\newcommand{\s}{\star}
\newcommand{\Sc}{\mathcal{S}}
\newcommand{\alb}{\bar{\al}}
\newcommand{\nsg}{\vartriangleleft}
\newcommand{\idl}{\vartriangleleft}
\newcommand{\Q}{\mathbb{Q}}
\newcommand{\F}{\mathbb{F}}
\newcommand{\norm}[1]{\left|\left| #1 \right|\right|}
\newcommand{\intR}[2]{\int_{\mathbb{R}} #1 \, d#2}
\newcommand{\E}{\mathbb{E}}
\newcommand{\eqsys}[1]{\begin{subequations}\algn{#1}\end{subequations}}
\newcommand{\eqsyslab}[2]{\begin{subequations} \label{eq:#2} \algn{#1}\end{subequations}}
\newcommand{\Fc}{\mathcal{F}}
\newcommand{\EE}[1]{\mathbb{E}\lrsq{#1}}
\newcommand{\EEC}[2]{\EE{\left. #1 \right| #2}}
\newcommand{\intab}{\int_{a}^{b}}
\newcommand{\DD}[2]{\frac{D #1}{D #2}}
\newcommand{\eps}{\epsilon}
\newcommand{\spd}{symmetric positive definite }
\newcommand{\NN}{\mathbb{N}}
\newcommand{\hch}{\hat{\chi}}
\newcommand{\hps}{\hat{\psi}}
\newcommand{\be}{\beta}
\newcommand{\TJi}[1]{\tensor{(J^{-1})}{#1}}
\newcommand{\Db}{\mathbb{D}}
\newcommand{\abs}[1]{\left| #1 \right|}
\newcommand{\lesim}{\lesssim}
\newcommand{\lessim}{\lesssim}
\newcommand{\lamdba}{\lambda}
\newcommand{\indnt}{\text{  }\text{  }\text{  }\text{  }\text{  }\text{  }\text{  }\text{  }}
\newcommand{\bbk}{\bar{\bar{\kappa}}}
\newcommand{\ok}{\tfrac{1}{\kappa}}
\newcommand{\vk}{\lrb{v,\tfrac{1}{\kappa}}}
\newcommand{\vkp}{\lrb{v',\tfrac{1}{\kappa}}}




\DeclareMathOperator{\diag}{diag}
\DeclareMathOperator{\Tr}{Tr}
\DeclareMathOperator{\Div}{div}
\DeclareMathOperator{\curl}{curl}
\DeclareMathOperator{\ad}{ad}
\DeclareMathOperator{\Ad}{Ad}
\DeclareMathOperator{\Orb}{Orb}
\DeclareMathOperator{\Stab}{Stab}
\DeclareMathOperator{\sgn}{sgn}
\DeclareMathOperator{\vol}{vol}
\DeclareMathOperator{\Span}{span}
\DeclareMathOperator{\sign}{sign}
\DeclareMathOperator{\Int}{Int}
\DeclareMathOperator{\area}{area}
\DeclareMathOperator{\hcf}{hcf}
\DeclareMathOperator{\lcm}{lcm}
\DeclareMathOperator{\Syl}{Syl}
\DeclareMathOperator{\im}{Im}
\DeclareMathOperator{\Ann}{Ann}
\DeclareMathOperator{\Cov}{Cov}
\DeclareMathOperator{\mse}{mse}
\DeclareMathOperator{\Var}{Var}
\DeclareMathOperator{\Diff}{Diff}
\DeclareMathOperator{\supp}{supp}
\DeclareMathOperator{\cn}{cn}
\DeclareMathOperator{\sn}{sn}
\DeclareMathOperator{\dn}{dn}
\DeclareMathOperator{\cosech}{cosech}
\DeclareMathOperator{\sech}{sech}





\theoremstyle{definition}
\newtheorem{defn}{Definition}[section] 
\newtheorem{examp}[defn]{Example}
\theoremstyle{plain}
\newtheorem{lem}[defn]{Lemma}
\newtheorem{thm}[defn]{Theorem}
\newtheorem{cor}[defn]{Corollary}
\newtheorem{prop}[defn]{Proposition}
\newtheorem{conj}[defn]{Conjecture}

\newcommand{\Theoremproof}[2]{
\begin{thm}
#1 
\end{thm}
\begin{proof}
#2
\end{proof}
}
\newcommand{\Theoremnoproof}[1]{
\begin{thm}
#1 
\end{thm}
}
\newcommand{\Lemmaproof}[2]{
\begin{lem}
#1 
\end{lem}
\begin{proof}
#2
\end{proof}
}
\newcommand{\Lemmanoproof}[1]{
\begin{lem}
#1 
\end{lem}
}
\newcommand{\Corollaryproof}[2]{
\begin{cor}
#1 
\end{cor}
\begin{proof}
#2
\end{proof}
}
\newcommand{\Corollarynoproof}[1]{
\begin{cor}
#1 
\end{cor}
}
\newcommand{\Definition}[2]{
\begin{defn}[#1]
#2 
\end{defn}
}
\newcommand{\Propproof}[2]{
\begin{prop}
#1 
\end{prop}
\begin{proof}
#2
\end{proof}
}
\newcommand{\Propnoproof}[1]{
\begin{prop}
#1 
\end{prop}
}

\crefname{lem}{Lemma}{Lemmas}
\Crefname{lem}{Lemma}{Lemmas}
\crefname{thm}{Theorem}{Theorems}
\Crefname{thm}{Theorem}{Theorems}
\crefname{def}{Definition}{Definitions}
\Crefname{def}{Definition}{Definitions}
\crefname{examp}{Example}{Examples}
\Crefname{examp}{Example}{Examples}
\crefname{figure}{Figure}{Figures}%
\crefname{table}{Table}{Tables}



\newcommand{\makeplots}[8]{

\IfFileExists{./MESHES/#2_#3/results_std/plots/#1_allyears_#4_#5_layer_#8.png}{
\IfFileExists{./MESHES/#2_#3/results_std/plots/#1_allyears_#6_#7_layer_#8.png}{
	\begin{figure}[H]
		\centering
		\begin{subfigure}{0.45\textwidth}
  			\centering
 			 \includegraphics[width=0.9\linewidth]{./MESHES/#2_#3/results_std/plots/#1_allyears_#4_#5_layer_#8.png}
		\end{subfigure}%
		\begin{subfigure}{0.45\textwidth}
 			 \centering
 			 \includegraphics[width=0.9\linewidth]{./MESHES/#2_#3/results_std/plots/#1_allyears_#6_#7_layer_#8.png}
		\end{subfigure}
	\end{figure}
	}
	\text{ }
}



\IfFileExists{./MESHES/#2_#3/results_std/plots/#1_fluct_sp_#4_#5_layer_#8.png}{
\IfFileExists{./MESHES/#2_#3/results_std/plots/#1_fluct_sp_#6_#7_layer_#8.png}{
	\begin{figure}[H]
		\centering
		\begin{subfigure}{0.45\textwidth}
  			\centering
 			 \includegraphics[width=0.9\linewidth]{./MESHES/#2_#3/results_std/plots/#1_fluct_sp_#4_#5_layer_#8.png}
		\end{subfigure}%
		\begin{subfigure}{0.45\textwidth}
 			 \centering
 			 \includegraphics[width=0.9\linewidth]{./MESHES/#2_#3/results_std/plots/#1_fluct_sp_#6_#7_layer_#8.png}
		\end{subfigure}
	\end{figure}
}
	\text{ }
}

\IfFileExists{./MESHES/#2_#3/results_std/plots/#1_fluct_t_#4_#5_layer_#8.png}{
\IfFileExists{./MESHES/#2_#3/results_std/plots/#1_fluct_t_#6_#7_layer_#8.png}{
	\begin{figure}[H]
		\centering
		\begin{subfigure}{0.45\textwidth}
  			\centering
 			 \includegraphics[width=0.9\linewidth]{./MESHES/#2_#3/results_std/plots/#1_fluct_t_#4_#5_layer_#8.png}
		\end{subfigure}%
		\begin{subfigure}{0.45\textwidth}
 			 \centering
 			 \includegraphics[width=0.9\linewidth]{./MESHES/#2_#3/results_std/plots/#1_fluct_t_#6_#7_layer_#8.png}
		\end{subfigure}
	\end{figure}
}
	\text{ }
}


}

\newcommand{\layernotimeplot}[4]{

\IfFileExists{./MESHES/#2_#3/results_std/plots/#1_tav_layer_#4.png}{
	\begin{figure}[H]
		\centering
		 \includegraphics[width=0.45\linewidth]{./MESHES/#2_#3/results_std/plots/#1_tav_layer_#4.png}
	\end{figure}
}

}

\newcommand{\timenolayerplot}[7]{

\IfFileExists{./MESHES/#2_#3/results_std/plots/#1_strat_#4_#5.png}{
\IfFileExists{./MESHES/#2_#3/results_std/plots/#1_strat_#6_#7.png}{
	\begin{figure}[H]
		\centering
		\begin{subfigure}{0.45\textwidth}
  			\centering
 			 \includegraphics[width=0.9\linewidth]{./MESHES/#2_#3/results_std/plots/#1_strat_#4_#5.png}
		\end{subfigure}%
		\begin{subfigure}{0.45\textwidth}
 			 \centering
 			 \includegraphics[width=0.9\linewidth]{./MESHES/#2_#3/results_std/plots/#1_strat_#6_#7.png}
		\end{subfigure}
	\end{figure}
	}
		\text{ }
}

}

\newcommand{\nolayernotimeplot}[3]{

\IfFileExists{./MESHES/#2_#3/results_std/plots/#1_timeseries.png}{
	\begin{figure}[H]
		\centering
		 \includegraphics[width=0.45\linewidth]{./MESHES/#2_#3/results_std/plots/#1_timeseries.png}
	\end{figure}
}

\IfFileExists{./MESHES/#2_#3/results_std/plots/#1_tav.png}{
	\begin{figure}[H]
		\centering
		 \includegraphics[width=0.45\linewidth]{./MESHES/#2_#3/results_std/plots/#1_tav.png}
	\end{figure}
}

}


\newcommand{\makeplotsnolayers}[7]{

\IfFileExists{./MESHES/#2_#3/results_std/plots/#1_allyears_#4_#5.png}{
\IfFileExists{./MESHES/#2_#3/results_std/plots/#1_allyears_#6_#7.png}{
	\begin{figure}[H]
		\centering
		\begin{subfigure}{0.45\textwidth}
  			\centering
 			 \includegraphics[width=0.9\linewidth]{./MESHES/#2_#3/results_std/plots/#1_allyears_#4_#5.png}
		\end{subfigure}%
		\begin{subfigure}{0.45\textwidth}
 			 \centering
 			 \includegraphics[width=0.9\linewidth]{./MESHES/#2_#3/results_std/plots/#1_allyears_#6_#7.png}
		\end{subfigure}
	\end{figure}
	}
		\text{ }
}

\IfFileExists{./MESHES/#2_#3/results_std/plots/#1_fluct_sp_#4_#5.png}{
\IfFileExists{./MESHES/#2_#3/results_std/plots/#1_fluct_sp_#6_#7.png}{
	\begin{figure}[H]
		\centering
		\begin{subfigure}{0.45\textwidth}
  			\centering
 			 \includegraphics[width=0.9\linewidth]{./MESHES/#2_#3/results_std/plots/#1_fluct_sp_#4_#5.png}
		\end{subfigure}%
		\begin{subfigure}{0.45\textwidth}
 			 \centering
 			 \includegraphics[width=0.9\linewidth]{./MESHES/#2_#3/results_std/plots/#1_fluct_sp_#6_#7.png}
		\end{subfigure}
	\end{figure}
	}
		\text{ }
}

\IfFileExists{./MESHES/#2_#3/results_std/plots/#1_fluct_t_#4_#5.png}{
\IfFileExists{./MESHES/#2_#3/results_std/plots/#1_fluct_t_#6_#7.png}{
	\begin{figure}[H]
		\centering
		\begin{subfigure}{0.45\textwidth}
  			\centering
 			 \includegraphics[width=0.9\linewidth]{./MESHES/#2_#3/results_std/plots/#1_fluct_t_#4_#5.png}
		\end{subfigure}%
		\begin{subfigure}{0.45\textwidth}
 			 \centering
 			 \includegraphics[width=0.9\linewidth]{./MESHES/#2_#3/results_std/plots/#1_fluct_t_#6_#7.png}
		\end{subfigure}
	\end{figure}
	}
		\text{ }
}

}

\newcommand{\allplot}[9]{

\makeplots{unod}{#1}{#2}{#3}{#4}{#5}{#6}{#7}
\makeplots{unod}{#1}{#2}{#3}{#4}{#5}{#6}{#8}
\makeplots{unod}{#1}{#2}{#3}{#4}{#5}{#6}{#9}
\layernotimeplot{unod}{#1}{#2}{#7}
\layernotimeplot{unod}{#1}{#2}{#8}
\layernotimeplot{unod}{#1}{#2}{#9}
\timenolayerplot{unod}{#1}{#2}{#3}{#4}{#5}{#6}
\nolayernotimeplot{unod}{#1}{#2}

\newpage
\makeplots{vnod}{#1}{#2}{#3}{#4}{#5}{#6}{#7}
\makeplots{vnod}{#1}{#2}{#3}{#4}{#5}{#6}{#8}
\makeplots{vnod}{#1}{#2}{#3}{#4}{#5}{#6}{#9}
\layernotimeplot{vnod}{#1}{#2}{#7}
\layernotimeplot{vnod}{#1}{#2}{#8}
\layernotimeplot{vnod}{#1}{#2}{#9}
\timenolayerplot{vnod}{#1}{#2}{#3}{#4}{#5}{#6}
\nolayernotimeplot{vnod}{#1}{#2}

\newpage
\makeplots{w}{#1}{#2}{#3}{#4}{#5}{#6}{#7}
\makeplots{w}{#1}{#2}{#3}{#4}{#5}{#6}{#8}
\makeplots{w}{#1}{#2}{#3}{#4}{#5}{#6}{#9}
\layernotimeplot{w}{#1}{#2}{#7}
\layernotimeplot{w}{#1}{#2}{#8}
\layernotimeplot{w}{#1}{#2}{#9}
\timenolayerplot{w}{#1}{#2}{#3}{#4}{#5}{#6}
\nolayernotimeplot{w}{#1}{#2}

\newpage
\makeplots{temp}{#1}{#2}{#3}{#4}{#5}{#6}{#7}
\makeplots{temp}{#1}{#2}{#3}{#4}{#5}{#6}{#8}
\makeplots{temp}{#1}{#2}{#3}{#4}{#5}{#6}{#9}
\layernotimeplot{temp}{#1}{#2}{#7}
\layernotimeplot{temp}{#1}{#2}{#8}
\layernotimeplot{temp}{#1}{#2}{#9}
\timenolayerplot{temp}{#1}{#2}{#3}{#4}{#5}{#6}
\nolayernotimeplot{temp}{#1}{#2}

\newpage
\makeplots{rhof}{#1}{#2}{#3}{#4}{#5}{#6}{#7}
\makeplots{rhof}{#1}{#2}{#3}{#4}{#5}{#6}{#8}
\makeplots{rhof}{#1}{#2}{#3}{#4}{#5}{#6}{#9}
\layernotimeplot{rhof}{#1}{#2}{#7}
\layernotimeplot{rhof}{#1}{#2}{#8}
\layernotimeplot{rhof}{#1}{#2}{#9}
\timenolayerplot{rhof}{#1}{#2}{#3}{#4}{#5}{#6}
\nolayernotimeplot{rhofmid}{#1}{#2}

\newpage
\makeplots{ke}{#1}{#2}{#3}{#4}{#5}{#6}{#7}
\makeplots{ke}{#1}{#2}{#3}{#4}{#5}{#6}{#8}
\makeplots{ke}{#1}{#2}{#3}{#4}{#5}{#6}{#9}
\layernotimeplot{ke}{#1}{#2}{#7}
\layernotimeplot{ke}{#1}{#2}{#8}
\layernotimeplot{ke}{#1}{#2}{#9}
\timenolayerplot{ke}{#1}{#2}{#3}{#4}{#5}{#6}
\nolayernotimeplot{ke}{#1}{#2}

\newpage
\makeplotsnolayers{ssh}{#1}{#2}{#3}{#4}{#5}{#6}
\timenolayerplot{ssh}{#1}{#2}{#3}{#4}{#5}{#6}
\nolayernotimeplot{ssh}{#1}{#2}

}

\newcommand{\momplot}[3]{

\IfFileExists{./DATA/513/#1_#2_#3.png}{
	\begin{figure}[H]
		\centering
		 \includegraphics[width=\linewidth]{./DATA/513/#1_#2_#3.png}
	\end{figure}
}

}

\newcommand{\momplottav}[3]{

\IfFileExists{./DATA/513/#1_tav_#2_#3.png}{
	\begin{figure}[H]
		\centering
		 \includegraphics[width=\linewidth]{./DATA/513/#1_tav_#2_#3.png}
	\end{figure}
}

}

\newcommand{\mitgcmplot}[3]{

\IfFileExists{../../../../../Documents/MITgcm/verification/tutorial_baroclinic_gyre/results/plots/#1_#2_#3.png}{
	\begin{figure}[H]
		\centering
		 \includegraphics[width=0.7\linewidth]{../../../../../Documents/MITgcm/verification/tutorial_baroclinic_gyre/results/plots/#1_#2_#3.png}
	\end{figure}
}

}


\begin{document}
\title{Differential Geometry: Definitions}
\date{}
\maketitle

\section{Manifolds, Vector fields, $k$-forms}
\defn{[Manifold]
A set $\M$ is an $n$-dimensional \i{smooth manifold} if it satisfies the following properties:
\begin{enumerate}[(i).]
	\item $\M$ is a topological space.
	\item There exist open sets $\O_{\al}$, which cover $\M$, and homeomorphisms (called \i{charts}) $\phi_{\al}: \O_{\al} \arr  U_{\al}$ such that the image $U_{\al}$ of $\phi_{\al}$ is an open subset of $\R^{n}$.
	\item If $\O_{\al}\cap \O_{\be}\neq \emptyset$ then the map $\phi_{\al}\circ \phi_{\be}^{-1}: \phi_{\be}(\O_{\al}\cap \O_{\be}) \arr \phi_{\al}(\O_{\al}\cap \O_{\be})$.
\end{enumerate}
}
\defn{[Smooth function]
A function $f:\M\arr \R$ is said to be \i{smooth} if for every chart $\phi_{\al}$ the function $F_{\al} := f \circ \phi_{\al}^{-1} : U_{\al} \arr \R $ is smooth. 
}\\
\linebreak
In practice we shall not distinguish between $f$ and $F_{\al}$, and we shall simply write $f(x)$ and assume that the chart is implicit. 
\defn{[Curve on a manifold]
A \i{curve on a manifold} is a map $\gamma : \R \arr \M$. $\gamma$ is \i{smooth} if for every chart $\phi_{\al}$, the curve $\phi_{\al}\circ \gamma : \R \arr U_{\al}$ is smooth.  
} 
\defn{[Tangent vector to a curve on a manifold]
Let $\gamma$ be a smooth curve of a manifold $\M$ such that $\gamma(0) = p \in \M$. The \i{tangent vector} $X_{p}$ to the curve $\gamma$ at the point $p$ is a map $\C^{\infty}(\M) \arr \R$ defined by:
\eqnn{
	X_{p}(f) = \left. \D{}{t}\right|_{t=0} f(\gamma(t)) = X^{i}_{p}\PD{f}{x^{i}}(p) = \B{X}_{p}\cdot \nabla f (p)
}
where  $X_{p}^{i} = \dot{\gamma}(0)^{i}$ and we denote $\B{X}_{p} = (X_{p}^{1},...,X_{p}^{n})$ to be the components of  $X_{p}$.
}
\defn{[Tangent space at the point $p$]
Let $p\in\M$. The \i{tangent space} to $\M$ a the point $p$, which we denote $T_{p}\M$, is defined to be the set of all tangent vectors at the point $p$. $T_{p}\M$ is a vector space isomorphic to $\R^{n}$, and its basis vectors are given, in a particular coordinate chart, by $\left.\PD{}{x^{i}}\right|_{p}$.
}
\defn{[Tangent bundle on $\M$]
The tangent bundle to $\M$ is:
\eqnn{
	T\M = \lrc{(p,V) : p\in \M,  V\in T_{p}\M}
}
}
\defn{[Vector field on $\M$]
A vector field on $\M$ is a map which takes points $p\in\M$ to vectors $X_{p}\in T_{p}\M$. We may represent a vector field in terms of the coordinate basis as:
\eqnn{
	X = X^{i} \PD{}{x^{i}}
}
This acts on functions on $\M$ as a directional derivative:
\eqnn{
	X(f)(x) = X^{i}(x)\PD{f}{x^{i}}(x)
}
A vector field is \i{smooth} if $X(f)$ is a smooth function on $\M$ for all smooth functions $f$. \\
The space of smooth vector fields on $\M$ is denoted $\mathfrak{X}(\M)$.
}
\defn{[Flow on $\M$]
	A collection of maps $g_{t}: \M \arr \M$, for $t\in\R$, is called a \i{flow} if:
	\begin{enumerate}
		\item $g_{t}$ is a diffeomorphism for all $t$. That is, a smooth map with a smooth inverse.
		\item $g_{0} = I$ for all $x\in \M$.
		\item $g_{t}\circ g_{s} = g_{t+s}$.
	\end{enumerate}	 
}
Note that there is a one-to-one correspondence between flows and vector fields. i.e. given a vector field $X$ we may define a flow $g_{t}x_{0}$ to be the solution to the following differential equation:
\eqnn{
	\D{}{t} x^{i}(t) = X^{i}(x(t)) \qquad x(0) = x_{0}
}
Conversely, given a flow $g_{t}$ we can define a vector field $X$ by, for $x\in \M$:
\eqnn{
	X(f)(x) = \left.\D{}{t}\right|_{t=0} f(g_{t}x) 
}
\defn{[Cotangent Space]
Given a tangent space $T_{p}\M$, we define its dual $T^{*}_{p}\M$ to be the set of all linear functions $T_{p}\M\arr\R$. $T^{*}_{p}\M$ is called the \i{cotangent space}. This is also a vector space isomorphic to $\R^{n}$ and, given a particular coordinate representation, we can define basis vectors $dx^{i}$ by $dx^{i}\lrb{\PD{}{x^{j}}} = \delta^{i}_{j}$, where $\PD{}{x^{j}}$ are the basis vectors for $T_{p}\M$. 
}
\defn{[One-forms]
	A \i{one-form} is a map which takes points $p\in\M$ to elements $T^{*}_{p}\M$. We can represent a general one-form in terms of the coordinate basis as:
	\eqnn{
		\al = \al_{i}dx^{i}
	}
	Then there is a natural pairing between one-forms and vector fields:
	\eqnn{
		\al(X)(x) = \al_{i}(x)dx^{i}\lrb{X^{j}(x)\PD{}{x^{j}}} = \al_{i}(x)X^{j}(x)\delta^{i}_{j} = \al_{i}(x)X^{i}(x) = \bs{\al}\cdot\B{X}
	}
	The space of one-forms is denoted by $\Lambda^{1}(\M)$
}
\defn{[Gradient]
	Any smooth function $f:\M\arr\R$ defines a one form via:
		\eqnn{
			df := \PD{f}{x^{i}}dx^{i}
		}
		this one form is called the \i{gradient} of $f$.
}
\defn{[Tensor]
	An $(r,s)$ \i{tensor} $T$ is a multi-linear map $T: (T_{p}^{*}\M)^{r}\otimes (T_{p}\M)^{s} \arr \R$. 
	\linebreak
	A basis for $(r,s)$ tensors is given by $ \PD{}{x^{j_{1}}} \otimes ... \otimes \PD{}{x^{j_{r}}} \otimes dx^{i_{1}}\otimes ... \otimes dx^{i_{s}}$ and in this basis $T$ has components $\tensor{T}{^{i_{1}...i_{r}}_{j_{1}...j_{s}}}$.\\
	\linebreak
	Thus, given $\al_{1},...,\al_{r}\in T_{p}^{*}\M$ and $X_{1},...,X_{s}\in T_{p}^{\M}$, we have:
	\eqnn{ 
		T(\al_{1},...,\al_{r},X_{1},...,X_{s})  = \tensor{T}{^{i_{1}...i_{r}}_{j_{1}...j_{s}}}(\al_{1})_{i_{1}},...,(\al_{r})_{i_{r}}X_{1}^{j_{1}}...X_{s}^{j_{s}}
	}
	We can define an $(r,s)$ \i{tensor field} to be a map from points $p\in\M$ to $(r,s)$ tensors.
}
\section{The Lie Derivative}
\defn{ [Pull-back of a function]
	Let $\psi : \M\arr \N$ be a diffeomorphism. Suppose $f:\N\arr\R$ is a smooth function on $\N$. We want to `pull $f$ back' so that we may view it as a function on $\M$. We therefore define the \i{pull-back} of $f$ by $\psi$ to be:
	\eqnn{
		(\psi{^{*}}f)(x) = f\circ \psi (x)
	}
}
\defn{ [Push-forward of a vector field]
	Let $X$ be a vector field on $\M$. We define the \i{push-forward} of $X$ by $\psi$ to be a vector field on $\N$ given by, for functions $f$ on $\N$:
		\eqnn{
			(\psi_{*}X)(f) = X(\psi^{*}(f))\circ \psi^{-1}
		}
}
In terms of components and denoting $y = \psi(x)$, we have:
\eqnn{
	(\psi_{*}X)(f)(y) = (\psi_{*}X)^{i}(y)\PD{f}{y^{i}}(y) = X^{i}(x)\PD{f\circ\psi}{x^{i}} (x) = X^{i}(x)\PD{\psi^{j}}{x^{i}}(x)\PD{f}{y^{j}}(y)
}
Thus $(\psi_{*}X)^{i} = \lrb{X^{j}\PD{\psi^{i}}{x^{j}}}\circ \psi^{-1}$.
\defn{[Pull-back of a one-form]
	Let $\al$ be a one-form on $\N$. We define the \i{pull-back} of $\al$ by $\psi$ by, for any vector field $X$ on $\M$:
	\eqnn{
		(\psi^{*}\al)(X) = \al(\psi_{*}X)\circ \psi
	}
}
We can similarly define the pull-back of a vector field and the push-forward of a one-form by using $\psi^{-1}$ instead of $\psi$. I.e. $(\psi^{-1})_{*} = \psi^{*}$ and $(\psi^{-1})^{*} = \psi_{*}$. Moreover, we can extend the definition to apply to any tensor field. That is, for an $(r,s)$ tensor $T$ defined on $\N$ we can define the pullback by:
\eqnn{
	(\psi^{*}T)(\al_{1},...,\al_{r},X_{1},...,X_{s}) = T(\psi_{*}\al_{1},...,\psi_{*}\al_{r}, \psi_{*}X_{1},...,\psi_{*}X_{s})\circ \psi
}
\defn{[Lie derivative]
	Let $g_{t}$ be a flow on $\M$ with corresponding vector field $X$. Let $T$ be a tensor field on $\M$. We define $\L_{X}T$, the \i{Lie derivative} of $T$ along the vector field $X$ by:
	\eqnn{
		\L_{X}T = \left.\D{}{t}\right|_{t=0} (g_{t}^{*}T)
	}
}
Note that the Lie derivative of an $(r,s)$ tensor field is also an $(r,s)$ tensor field. \\
For example, the Lie derivative along $X$ of a vector field $Y$ is given by:
\algnn{
	(\L_{X}Y)(f)(x) & = \left.\D{}{t}\right|_{t=0} (g_{t}^{*}Y)(f)(x) \\
					  & =  \left.\D{}{t}\right|_{t=0}  Y(f\circ g_{t}^{-1})(g_{t}x)  \\
					  & = \left.\D{}{t}\right|_{t=0} Y^{i}(g_{t}x)\PD{f\circ g_{t}^{-1}}{x^{i}}(g_{t}x)  \\
					  & = \left.\D{}{t}\right|_{t=0}Y^{i}(g_{t}x)\PD{f}{x^{j}}(x)\PD{(g_{t}^{-1})^{j}}{x^{i}}(g_{t}x) \\
					   & = \left.\D{}{t}\right|_{t=0}Y^{i}(g_{t}x)\PD{f}{x^{j}}(x)\lrsq{\lrsq{\nabla g_{t}(x)}^{-1}}^{j}_{i} \\
					  & = \left[(\dot{g}_{t}x)^{j}  \PD{Y^{i}}{x^{j}}(g_{t}x) \PD{f}{x^{i}}(x)\lrsq{\lrsq{\nabla g_{t}(x)}^{-1}}^{j}_{i}  - Y^{i}(x) \PD{f}{x^{j}}(x)\lrsq{\lrsq{\nabla g_{t}(x)}^{-1}}^{j}_{k}\lrsq{\nabla \dot g_{t}(x)}^{k}_{l}\lrsq{\lrsq{\nabla g_{t}(x)}^{-1}}^{l}_{i} \right]_{t=0} \\
					  & = X^{j}  \PD{Y^{i}}{x^{j}}(x) \PD{f}{x^{i}}(x) - Y^{i}(x) \PD{f}{x^{j}}(x)\PD{X^{j}}{x^{i}}(x) \\
					  & = \lrsq{X,Y}f(x)
}
Where we have used the fact that:
\algnn{
	\delta^{i}_{j} & = \PD{x^{i}}{x^{j}} = \PD{}{x^{j}} \lrsq{g_{t}^{-1}(g_{t}x)}^{i} \\
					& = \lrsq{\PD{g_{t}^{-1}}{x^{k}}(g_{t}x)}^{i}\lrsq{\PD{g_{t}}{x^{j}}(x)}^{k}\\
					& = \lrsq{\nabla g_{t}^{-1} (g_{t}x)}^{i}_{k}\lrsq{\nabla g_{t}(x)}^{k}_{j}\\
					\implies\qquad &\PD{(g_{t}^{-1})^{j}}{x^{i}}(g_{t}x) = \lrsq{\lrsq{\nabla g_{t}(x)}^{-1}}^{j}_{i}
}
and that for a matrix $A(t)$, we have $\D{}{t} A^{-1}(t) = - A^{-1}(t)\dot{A}(t)A^{-1}(t)$. \\
We conclude that:
\eqnn{
	\L_{X}Y = [X,Y]
}
That is, the Lie derivative of a vector field is given by the \i{vector field commutator}, which is explicitly defined by:
\eqnn{
	[X,Y] = \lrb{X^{i}\PD{Y^{j}}{x^{i}} - Y^{i}\PD{X^{j}}{x^{i}}}\PD{}{x^{i}}
}

\section{Group actions on Manifolds}
\defn{[Group Action]
	Let $G$ be a group. $G$ is said to \i{act on} $M$ if there exists a map $\Phi : G\times \M \arr$ such that:
	\begin{enumerate}[(i).]
		\item If $e$ is the identity element of $G$ then $\Phi(e,p) = p$ for all $p\in \M$. 
		\item For $g_{1},g_{2}\in G$ we have $\Phi(g_{1}, \Phi(g_{2},p)) = \Phi(g_{1}g_{2},p)$.
	\end{enumerate}
	Usually we do not write the group action in terms of $\Phi$ and simply denote $gp = g(p) := \Phi(g,p)$. Then the second condition above becomes $g_{1}(g_{2}p) = (g_{1}g_{2})p$.  
}
Effectively this definition means that every element of $G$ defines a map from the manifold $\M$ to itself. 
\defn{[Lie Group]
	A \i{Lie group} is a group that is also a manifold.
}\\
\linebreak
Note that the space of all diffeomorphisms on $\M$ forms a Lie group, albeit an infinite-dimensional one. A flow is a smooth curve on the space of diffeomorphisms. 
\defn{[Lie algebra of a Lie group]
	If $G$ is a Lie group then the tangent space to $G$ at the identity element is called the \i{Lie algebra} of $G$ and this is denoted $\mathfrak{g} = T_{e}G$. 
}
\section{The Diamond Operator}
If $G=\text{Diff}(\M)$, the space of all diffeomorphisms on $\M$, then its Lie algebra is the space of vector fields on $\M$. \\
\defn{[Diamond operator]
	Let $V$ be some space of tensors on $\M$ with dual $V^{*}$. Let $ \mathfrak{X}(\M)$ be the space of vector fields on $\M$. Then for $p\in V^{*}$ and $q\in V$ we define the \i{diamond} $p\diamond q \in \mathfrak{X}^{*}(\M)$ such that for all $\xi\in\mathfrak{X}(\M)$
	\eqnn{
		\EXP{p\diamond q, \xi}_{\mathfrak{X}^{*}(\M)\times\mathfrak{X}^{*}(\M)} = \EXP{p,-\L_{\xi}q}_{V^{*}\times V}
	}
}

\section{Exterior Derivative}
We have already defined the gradient of a function $df (x) = \PD{f}{x^{i}}(x)dx^{i}$, and this is given by a 1-form. We can generalise the definition of the operator $d$ as follows. 
\defn{[$k$-form]
	A $k$-form is an antisymmetric $(0,k)$ tensor. On an $n$-dimensional manifold a basis for $k$-forms ($k\leq n$) is given by:
	\eqnn{
		dx^{i_{1}}\wedge ... \wedge dx^{i_{k}}
	}
	where the \i{wedge product} is defined by:
	\eqnn{
		dx\wedge dy = 2\lrb{dx\otimes dy - dy\otimes dx}
	}
	And a general $k$-form is written as:
	\eqnn{
		\al = \frac{1}{k!}\al_{i_{1}...i_{k}}dx^{i_{1}}\wedge ... \wedge dx^{i_{k}}
	}
	In terms of the basis for $(0,k)$ tensors this is:
	\eqnn{
		\al = \al_{i_{1}...i_{k}}dx^{i_{1}}\otimes ... \otimes dx^{i_{k}}
	}
	where $\al_{i_{1}...i_{k}}$ is antisymmetric in its indices.
}
\defn{[Exterior derivative]
	Let $\al\in \Lambda^{k}(\M)$ be a $k$-form. We define the \i{exterior derivative} $d:\Lambda^{k}(\M) \arr \Lambda^{k+1}(\M)$ by:
	\algnn{
		d\al & = d\lrb{\frac{1}{k!}\al_{i_{1}...i_{k}}dx^{i_{1}}\wedge...\wedge dx^{i_{k}}} \\
				& = \frac{1}{k!}\pr_{j}\al_{i_{1}...i_{k}}dx^{j}\wedge dx^{i_{1}}\wedge...\wedge dx^{i_{k}}
	}
}
Note that $d^{2} = 0$. \\
\linebreak
Furthermore, we have Cartan's formula:
\eqnn{
	\L_{X}\al = i_{X}(d\al) + d(i_{X}\al) 
}
where $i_{X}\al = \al(X, \cdot , ..., \cdot)$ is a $(k-1)$-form if $\al$ is s $k$-form.  \\
\linebreak
From this we can prove another important property, namely that $d$ commutes with the Lie Derivative:
\eqnn{
	d(\L_{X}\al) = d(i_{X}d\al) = \L_{X}d\al
}
\section{Integration on Manifolds}

\defn{[Integration]
Let $\O\subset\M$ have a chart $\psi : \O \arr \R^{n}$. Let $\omega$ be an $n$-form on $\M$. We define the \i{integral} of $\omega$ over $\O$ to be:
\eqnn{
	\int_{\O} \omega = \int_{\psi(\O)}\omega_{1...n}(\psi^{-1}(x))dx^{1}...dx^{n}
}
\defn{[Embedding]
Let $\M$, $\N$ be manifolds of dimension $m,n$ respectively with $m<n$. An \i{embedding} of $\M$ into $\N$ is a smooth, one-to-one map $\phi:\M\arr\N$.
}
We may therefore evaluate the following integral, for an $m$-form $\eta$ defined on $\N$:
\eqnn{
	\int_{\phi(\M)}\eta = \int_{\M}\phi^{*}\eta
}
}
\end{document}
